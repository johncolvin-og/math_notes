\documentclass{article}
\usepackage{array}
\usepackage{booktabs}
\usepackage{amsmath}
\usepackage{amssymb}
\usepackage{mathtools}
\usepackage{stackrel}
\usepackage[dvipsnames,definecolor]{xcolor}
\usepackage{array}
\usepackage{amsmath}
\usepackage{amssymb}
\usepackage{mathtools}
\usepackage[T1]{fontenc}
\usepackage{stackrel}
\usepackage[colorlinks]{hyperref}
\usepackage[user,titleref]{zref}
\usepackage[dvipsnames,definecolor]{xcolor}
\usepackage[most]{tcolorbox}
\usepackage{scrextend}

\usepackage{enumitem}
\usepackage{framed}
\usepackage{multicol}

\usepackage{xcolor}
\colorlet{linkequation}{red}
\colorlet{linktheo}{green}
\usepackage[colorlinks]{hyperref}

\definecolor{propbg}{RGB}{180,220,255}
\definecolor{factbg}{RGB}{170,232,145}
\definecolor{formula_header}{RGB}{30,40,210}

\newcounter{theoremc}
\newcommand{\theoremc}{\refstepcounter{theoremc}{\noindent\textbf{Theorem} \arabic{theoremc}}}
\newcommand{\myparbox}{\parbox[p]{12cm}}
\newcommand{\vpush}{\vspace{3mm}}
\newcommand{\sectionbreak}{\makebox[\linewidth]{\rule{12cm}{0.4pt}}}
\newcommand{\minmax}[2]{\min\limits_{#1}~\max\limits_{#2}}
\newcommand{\maxmin}[2]{\max\limits_{#1}~\min\limits_{#2}}

\newenvironment{note}[1][]{
  \textbf{#1}
  \begin{flushleft}
  \begin{addmargin}[2em]{0em}
}{
  \end{addmargin}
  \end{flushleft}
}
\newenvironment{proof}
{
\noindent
\begin{addmargin}[2em]{0em}
\textbf{Proof}
}
{
\end{addmargin}
}

\newenvironment{headered_note}[1][]{
\begin{flushleft}
\begin{addmargin}[2em]{0em}
\textbf{#1\\}
\vpush
\begin{addmargin}[2em]{0em}
}
{
\end{addmargin}
\end{addmargin}
\end{flushleft}
}

\newenvironment{fact}[1][]{
\begin{flushleft}
\begin{addmargin}[2em]{0em}
\textbf{Fact: #1\\}
\vpush
\begin{addmargin}[2em]{0em}
}
{
\end{addmargin}
\end{addmargin}
\end{flushleft}
}

\newenvironment{theorem}[1][]{
\begin{tcolorbox}[colback=green!30,%gray background
colframe=black,% black frame colour
% 5cm,% Use 5cm total width,
arc=3mm, auto outer arc,
]
\theoremc: \space \textbf{#1}
\begin{addmargin}[2em]{0em}
}
{
\end{addmargin}
\end{tcolorbox}}

\newenvironment{formula}[1]{
\begin{tcolorbox}[colback=white,%gray background
colframe=black,% black frame colour
% 5cm,% Use 5cm total width,
arc=3mm, auto outer arc,
]
\begin{flushleft}
{\color{formula_header} \textbf{Formula}}
\textbf{#1}
\end{flushleft}
\begin{addmargin}[2em]{0em}
}
{
\end{addmargin}
\end{tcolorbox}}

\newenvironment{headered_box}[1]{
\begin{tcolorbox}[colback=white,%gray background
colframe=black,% black frame colour
% 5cm,% Use 5cm total width,
arc=3mm, auto outer arc,
]
\begin{flushleft}
\textbf{#1}
\end{flushleft}
\begin{addmargin}[2em]{0em}
}
{
\end{addmargin}
\end{tcolorbox}}

\newenvironment{red_rounded_box}{
\begin{tcolorbox}[enhanced jigsaw,
    colback=gray!30,%gray background
    colframe=Maroon,% black frame colour
    width=5cm,% Use 5cm total width,
    arc=3mm, auto outer arc,
    boxrule=5pt,
    drop shadow={Maroon!50!gray!80}
  ]
}
{ \end{tcolorbox} }

\newcommand*{\reftheo}[1]{%
  \begingroup
    \hypersetup{
      linkcolor=linktheo,
      linkbordercolor=linktheo,
    }%
    (\ref{#1})%
  \endgroup
}

\newcommand*{\refequation}[1]{%
  \begingroup
    \hypersetup{
      linkcolor=linkequation,
      linkbordercolor=linkequation,
    }%
    \ref{#1}%
  \endgroup
}

\newenvironment{axiom}{
  \begin{tcolorbox}[enhanced jigsaw,
    colback={blue!20},
    colframe=Maroon,
    arc=2mm, auto outer arc,
    boxrule=3pt,
    drop shadow={blue!60!gray!60}
  ]
}{
  \end{tcolorbox}
}
\newenvironment{proposition}[2]{
\noindent\fcolorbox{white}{propbg}{%
  \parbox{\textwidth}{%
    \textbf{Proposition #1}
    {#2}
  }
}
}

\newcounter{definitionno}
\newenvironment{definition}[1][]{
  \begin{flushleft}
  \begin{addmargin}[2em]{0em}
  \text{Definition \refstepcounter{definitionno}\arabic{definitionno}}
  \textbf{#1}\\
  \vpush
  \begin{addmargin}[2em]{0em}
}{
  \end{addmargin}
  \end{addmargin}
  \end{flushleft}
}

\newcounter{sectiondefno}
\newenvironment{sectiondef}[1]{
  \begin{flushleft}
  \begin{addmargin}[2em]{0em}
  \textbf{Definition \thesection.\refstepcounter{sectiondefno}\arabic{sectiondefno}}
  \textbf{#1}\\
  \vpush
  \begin{addmargin}[2em]{0em}
}{
  \end{addmargin}
  \end{addmargin}
  \end{flushleft}
}

\makeatletter
\newcommand*{\inlineequation}[2][]{%
  \begingroup
    % Put \refstepcounter at the beginning, because
    % package `hyperref' sets the anchor here.
    \refstepcounter{equation}%
    \ifx\\#1\\%
    \else
      \label{#1}%
    \fi
    % prevent line breaks inside equation
    \relpenalty=10000 %
    \binoppenalty=10000 %
    \ensuremath{%
      % \displaystyle % larger fractions, ...
      #2%
    }%
    ~\@eqnnum
  \endgroup
}
\makeatother

\setlength{\fboxsep}{12pt}


\title{Convex Optimization 725 Lecture 2: Convexity I: Sets and Functions}
\date{2021-02-21}
\author{John Colvin}

\begin{document}
\maketitle

\begin{definition}[Convex Hull]
  For a given set $C$ (not necessarily convex), the intersection of all convex
  sets that contain $C$ is known is the \textbf{Convex hull of C}.  In other words,
  the convex hull is \textit{smallest} set that contains every convex
  combination of points in $C$.\\
  \vpush
  To reiterate, the convex hull contains \textbf{all of C}.  If $C$ is not already
  convex, this means the \textbf{convex hull} of $C$ must fill in all gaps between
  points in $C$ that make $C$ non-convex.
\end{definition}
\begin{definition}[Closed Set]
  \begin{itemize}
    \item a set that contains all of its limit points.
    \item a set whose complement is \textbf{Open}
    \item a set that always contains points of convergence (if it is the domain)
  \end{itemize}
\end{definition}
\begin{headered_note}[Examples of Convex Sets]
  \begin{itemize}
    \item Trivial ones: empty set, point, line, plane, etc.
    \item Norm ball: $\{x~:~||x||\leq r\}$ for given norm $||~\dot||$, radius $r$
          \begin{itemize}
            \item True because norms satisfy the \textbf{triangle inequality}
            \item Norms have \textbf{Positive Homogeniety}
          \end{itemize}
    \item Hyperplane (mentioned in trivial): $\{x~:~a^Tx=b\}$ for given $a,b$
    \item Half space: $\{x~:~a^Tx\leq b\}$ for given $A,b$
    \item Affine space: $\{x~:~Ax=b\}$ for given $A,b$
  \end{itemize}
\end{headered_note}
\pagebreak
\begin{definition}[Polyhedron]
  An intersection of two or more \textbf{half spaces}.  The set of all $x$ that
  satisfy:
  \begin{align}
    \{x~:~Ax~\leq b\}
  \end{align}
  where the inequality means
  \begin{align}
    Ax\leq b \longleftrightarrow a_i^Tx_i\leq b_i~:~\forall i
  \end{align}
  The intersection of all half spaces $a_i^Tx\leq b_i$ (each $i$ represents a half space).\\
  \vpush
  \textit{Interesting Note}: the set $\{x~:~Ax\leq b,~Cx=d\}$ is also a polyhedron!
  This is so, because $Cx=d$ can be rewritten as 2 inequalities:
  \begin{align}
    \nonumber
    Cx  & \leq d  \\
    -Cx & \leq -d
  \end{align}
  Thus, by the original definition, the polyhedron is still an \textbf{intersection of halfspaces}.
\end{definition}
\begin{definition}[Simplex]
  A \textbf{polyhedron} whose members $\{x_1,...,x_k\}$ are linearly \textbf{indenpendent}
\end{definition}
\begin{theorem}[Separating Hyperplane Theorem]
  2 disjoint convex sets have a hyperplane between them.\\
  \vpush
  This makes sense, because the lack of a hyperplane between them means
  a hyperplane (or many) intersects one of them in (at least) two places, and
  another in (at least) one.  Then it would follow that
  a line (or many) intersects both of them.  Since they don't intersect each other,
  that line would not be contained in either set.
\end{theorem}
\begin{theorem}[Supporting Hyperplane Theorem]
  A boundary point of a convex set has a \textbf{supporting hyperplane}
  that is tangent to it (does not touch any other boundary point).\\
  \vpush
  Formally, if $C$ is a nonempty convex set, and $x_0\in dom(C)$ then
  $\exists A$ such that:
  \begin{align}
    C\subseteq \{x:a^Tx\leq a^tx_0\}
  \end{align}
\end{theorem}
\pagebreak
\begin{note}
  \textbf{Operations preserving concavity}
  \begin{itemize}
    \item the \textbf{Intersection} of convex sets is always convex
    \item \textbf{Scaling and translation:} if $C$ is convex, then
          \begin{align}
            aC+b=(ax+b:x\in C)
          \end{align}
          is convex for any a, b
    \item \textbf{Affine images and preimages}: Given an affine function
          $f(x)=Ax+b$ and convex set $C$, $f(C)$ is convex:
          \begin{align}
            f(C)=\{f(x):x\in C\}
          \end{align}
          if $D$ is convex then,
          \begin{align*}
            f^{-1}(D)=\{x:f(x)\in D\}
          \end{align*}
          is convex
  \end{itemize}
\end{note}
\sectionbreak
\begin{definition}[Linear Matrix Inequality]
  Symmetric matrix inequality: given two symmetric matrices $A$ and $B$,
  \begin{align}
    A\geq B\longleftrightarrow A-B\text{~is positive semidefinite}
  \end{align}
\end{definition}
\begin{headered_note}[Linear Matrix Inequality Solution Set]
  Given $A_1,...,A_k,B\in S^n$, a linear matrix inequality is of the form
  $x_1A_1+...+x_kA_k\preceq B$ for $x\in \mathbb{R}^k$, the set $C$ of points
  $x$ that satisfy the inequality is convex.
  \begin{align}
    \nonumber
    B-\sum_{i=1}^k(tx_i+(1-t)y_i)A~\text{is}~\textbf{positive semi definite} \\
    v^T\left(B-\sum_{i=1}^k (tx_i+(1-t)y_i)A\right)v\geq 0
  \end{align}
  \begin{headered_note}[Alt Proof]
    let $f:\mathbb{R}^k\rightarrow S^n,f(x)=B-\sum_{i=1}^kx_iA_i$, and note that
    \begin{align}
      C=f^{-1}(S^n)
    \end{align}
    is the affine preimage of a convex set (the \textbf{PSD cone}).  Remeber, the
    \textit{domain} of a convex function is a convex set.  Since the inverse
    function $f^{-1}(S^n)$ has an \textit{image} that is the \textit{domain of $f$},
    both are convex!
  \end{headered_note}
\end{headered_note}
\begin{fact}
  \begin{align}
    f\text{ concave }\Longleftrightarrow -f\text{ convex }
  \end{align}
\end{fact}
\begin{definition}[Affine Transformation]
  A geometric transformation that preserves lines and parallelism, but not
  necessarily distances and angles
\end{definition}

\begin{definition}[Strictly Convex]
  $f$ is convex, and has greater curvature than a linear function.
  \begin{align}
    f(tx+(1-t)y)<tf(x)+(1-t)f(y):\{x~|~x\neq y\},\{t~|~0<t<1\}
  \end{align}
\end{definition}
\begin{definition}[Strongly Convex]
  $f$ is at least as convex as a quadratic function.  In math terms, $f$ is
  strongly convex if the following is convex:
  \begin{align}
    f-{m\over 2}||x||^2_2:\{m~|~m>0\}
  \end{align}
\end{definition}
\begin{headered_note}[Examples of Convex Functions]
  \begin{itemize}
    \item Univariate functions:\\
          \vpush
          \begin{tabular}{ c | c | c | c | c }
            Exponential Fn & convex  & $e^{ax}$  & $\{x|x\in \mathbb{R}\}$      & $\{a|\forall a\}$             \\
            Power Fn       & convex  & $x^a$     & $\{x|x\in \mathbb{R}_+\}$    & $\{a|a\geq 1 \lor a \leq 0\}$ \\
            Power Fn       & concave & $x^a$     & $\{x|x\in \mathbb{R}_+\}$    & $\{a|0\leq a\leq 1\}$         \\
            Log Fn         & concave & $\log(x)$ & $\{x|x\in \mathbb{R}_{++}\}$ &                               \\
          \end{tabular}
    \item Affine Fn: $a^Tx+b$ is both convex and concave, as
          \begin{align}
            f(tx+(1-t)y)=tf(x)+(1-t)f(y)
          \end{align}
    \item Quadratic Fn: ${1\over 2}x^TQx+b^Tx+c$ is convex provided $Q$ is \textbf{positive semidefinite}
          \begin{itemize}
            \item This is the sum of two convex functions! $f(x)=PSD(x)+Linear(x)$
          \end{itemize}
    \item Least squares loss: $||y-Ax||^2_2$ is always convex (since $A^TA$ is \textbf{PSD} by definition)
          \begin{itemize}
            \item This function expands to a quadratic
                  \begin{align*}
                    f(x) & =\left(y-Ax\right)^T\left(y-Ax\right) \\
                         & =y^Ty-y^TAx-(Ax)^Ty+(Ax)^T(Ax)        \\
                         & =x^TA^TAx-2y^TAx+y^Ty                 \\
                         & =x^TSx-(2y^TA)x+c
                  \end{align*}
          \end{itemize}
  \end{itemize}
\end{headered_note}
\begin{headered_note}[Indicator Function]
  If $C$ is convex, then its indicator function is convex
  \begin{align}
    I_c(x)=\begin{cases}
      0      & x\in C    \\
      \infty & x\notin C
    \end{cases}
  \end{align}
\end{headered_note}
\begin{headered_note}[Support Function]
  For any set $C$ (convex or not), its support function is convex
  \begin{align}
    I_c(x)=\max_{y\in C}x^Ty
  \end{align}
\end{headered_note}
\begin{headered_note}[Max Function]
  Always convex
  \begin{align}
    f(x)=\max\{x_1,...,x_n\}
  \end{align}
\end{headered_note}
\begin{definition}[Epigraph Characterization]
  The closed set of points above $f(x)$.  In math terms:
  \begin{align}
    epi(f)=\{(x,t)\in dom(f)\times \mathbb{R}:f(x)\leq t\}
  \end{align}
  $f(x)$ is \textbf{convex} if and only if its \textbf{epigraph} is a \textbf{convex set}.
\end{definition}
\begin{headered_note}[Convex Sublevel Sets]
  if $f(x)$ is convex, then its sublevel sets
  \begin{align}
    \{x\in dom(f):f(x)\leq t\}
  \end{align}
  are convex $\forall t\in \mathbb{R}$.  The converse is not true.
\end{headered_note}
\begin{definition}[First-Order Characterization]
  if $f(x)$ is \textbf{differentiable}, the $f(x)$ is convex if and only if
  $dom(f)$ is convex, and
  \begin{align}
    f(y)\geq f(x)+\nabla f(x)^T(y-x)
  \end{align}
  This is the multivariable version of
  \begin{align}
    f(y)\geq f(x)+f^\prime(x)(y-x)
  \end{align}
  \textbf{My Thoughts:} \textit{y's accelaration must never be negative with respect to x}.
\end{definition}
\pagebreak
\begin{definition}[Second-Order Characterization]
  Given a twice differentiable function $f(x)$ with a convex domain $dom(f)$,
  $f(x)$ is convex if and only if
  \begin{align}
    \nabla^2f(x)\succeq 0:\{x|x\in dom(f)\}
  \end{align}
  Thus
  \begin{align}
    \nabla f(x)=0\Longleftrightarrow x minmizes f
  \end{align}
  This means the \textbf{Hessian} matrix of $f(x)$ must be \textbf{PSD}
\end{definition}
\begin{theorem}[Jensen's Inequality]
  If $f(x)$ is convex, and $X$ is a random variable supported on $dom(f)$, then
  \begin{align}
    f(E[X])\leq E[f(X)]
  \end{align}
\end{theorem}
\begin{headered_note}[Operations preserving Convexity]
  \begin{headered_note}[Non-negative Linear Combination]
    \begin{align}
      \stackrel{convex}{f_1,...,f_m}\implies \stackrel{convex}{a_1f_1+...+a_mf_m}:\{a_1|a_1\geq 0\}
    \end{align}
  \end{headered_note}
  \begin{headered_note}[Pointwise Maximization]
    \begin{align}
      \stackrel{convex}{f_s(s):\{s|s\in S\}}\implies \stackrel{convex}{\max_{s\in S}f_s(x)}
    \end{align}
  \end{headered_note}
  \begin{headered_note}[Partial Minimization]
    \begin{align}
      \stackrel{convex}{g(x,y)}\wedge \stackrel{convex}{C}\implies \stackrel{convex}{\min_{y\in C}g(x,y)}
    \end{align}
  \end{headered_note}
\end{headered_note}
\begin{headered_note}[Example: Max Distance to a Set]
  \begin{itemize}
    \item Let $C$ be an arbitrary set (not necessarily convex)
    \item Let $f(x)=\max_{y\in C}||x-y||$
    \item $f_y(x)=||x-y||$ is convex in $x$ for any fixed $y$, so by the
          \textbf{pointwise maximization rule}, $f$ is convex.
  \end{itemize}
\end{headered_note}
\begin{headered_note}[Example: Min Distance to a Set]
  \begin{itemize}
    \item Let $C$ be an arbitrary set (not necessarily convex)
    \item Let $f(x)=\min_{y\in C}||x-y||$
    \item $g(x,y)=||x-y||$ is convex in $x,y$ jointly, and $C$ is assumed
          convex, so apply the \textbf{partial minimization rule} to show
          that $\stackrel{convex}{C}\implies \stackrel{convex}{f(x)}$
  \end{itemize}
\end{headered_note}
\end{document}
