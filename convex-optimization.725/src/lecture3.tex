\documentclass{article}
\usepackage{array}
\usepackage{booktabs}
\usepackage{amsmath}
\usepackage{amssymb}
\usepackage{mathtools}
\usepackage{stackrel}
\usepackage[dvipsnames,definecolor]{xcolor}
\usepackage{array}
\usepackage{amsmath}
\usepackage{amssymb}
\usepackage{mathtools}
\usepackage[T1]{fontenc}
\usepackage{stackrel}
\usepackage[colorlinks]{hyperref}
\usepackage[user,titleref]{zref}
\usepackage[dvipsnames,definecolor]{xcolor}
\usepackage[most]{tcolorbox}
\usepackage{scrextend}

\usepackage{enumitem}
\usepackage{framed}
\usepackage{multicol}

\usepackage{xcolor}
\colorlet{linkequation}{red}
\colorlet{linktheo}{green}
\usepackage[colorlinks]{hyperref}

\definecolor{propbg}{RGB}{180,220,255}
\definecolor{factbg}{RGB}{170,232,145}
\definecolor{formula_header}{RGB}{30,40,210}

\newcounter{theoremc}
\newcommand{\theoremc}{\refstepcounter{theoremc}{\noindent\textbf{Theorem} \arabic{theoremc}}}
\newcommand{\myparbox}{\parbox[p]{12cm}}
\newcommand{\vpush}{\vspace{3mm}}
\newcommand{\sectionbreak}{\makebox[\linewidth]{\rule{12cm}{0.4pt}}}
\newcommand{\minmax}[2]{\min\limits_{#1}~\max\limits_{#2}}
\newcommand{\maxmin}[2]{\max\limits_{#1}~\min\limits_{#2}}

\newenvironment{note}[1][]{
  \textbf{#1}
  \begin{flushleft}
  \begin{addmargin}[2em]{0em}
}{
  \end{addmargin}
  \end{flushleft}
}
\newenvironment{proof}
{
\noindent
\begin{addmargin}[2em]{0em}
\textbf{Proof}
}
{
\end{addmargin}
}

\newenvironment{headered_note}[1][]{
\begin{flushleft}
\begin{addmargin}[2em]{0em}
\textbf{#1\\}
\vpush
\begin{addmargin}[2em]{0em}
}
{
\end{addmargin}
\end{addmargin}
\end{flushleft}
}

\newenvironment{fact}[1][]{
\begin{flushleft}
\begin{addmargin}[2em]{0em}
\textbf{Fact: #1\\}
\vpush
\begin{addmargin}[2em]{0em}
}
{
\end{addmargin}
\end{addmargin}
\end{flushleft}
}

\newenvironment{theorem}[1][]{
\begin{tcolorbox}[colback=green!30,%gray background
colframe=black,% black frame colour
% 5cm,% Use 5cm total width,
arc=3mm, auto outer arc,
]
\theoremc: \space \textbf{#1}
\begin{addmargin}[2em]{0em}
}
{
\end{addmargin}
\end{tcolorbox}}

\newenvironment{formula}[1]{
\begin{tcolorbox}[colback=white,%gray background
colframe=black,% black frame colour
% 5cm,% Use 5cm total width,
arc=3mm, auto outer arc,
]
\begin{flushleft}
{\color{formula_header} \textbf{Formula}}
\textbf{#1}
\end{flushleft}
\begin{addmargin}[2em]{0em}
}
{
\end{addmargin}
\end{tcolorbox}}

\newenvironment{headered_box}[1]{
\begin{tcolorbox}[colback=white,%gray background
colframe=black,% black frame colour
% 5cm,% Use 5cm total width,
arc=3mm, auto outer arc,
]
\begin{flushleft}
\textbf{#1}
\end{flushleft}
\begin{addmargin}[2em]{0em}
}
{
\end{addmargin}
\end{tcolorbox}}

\newenvironment{red_rounded_box}{
\begin{tcolorbox}[enhanced jigsaw,
    colback=gray!30,%gray background
    colframe=Maroon,% black frame colour
    width=5cm,% Use 5cm total width,
    arc=3mm, auto outer arc,
    boxrule=5pt,
    drop shadow={Maroon!50!gray!80}
  ]
}
{ \end{tcolorbox} }

\newcommand*{\reftheo}[1]{%
  \begingroup
    \hypersetup{
      linkcolor=linktheo,
      linkbordercolor=linktheo,
    }%
    (\ref{#1})%
  \endgroup
}

\newcommand*{\refequation}[1]{%
  \begingroup
    \hypersetup{
      linkcolor=linkequation,
      linkbordercolor=linkequation,
    }%
    \ref{#1}%
  \endgroup
}

\newenvironment{axiom}{
  \begin{tcolorbox}[enhanced jigsaw,
    colback={blue!20},
    colframe=Maroon,
    arc=2mm, auto outer arc,
    boxrule=3pt,
    drop shadow={blue!60!gray!60}
  ]
}{
  \end{tcolorbox}
}
\newenvironment{proposition}[2]{
\noindent\fcolorbox{white}{propbg}{%
  \parbox{\textwidth}{%
    \textbf{Proposition #1}
    {#2}
  }
}
}

\newcounter{definitionno}
\newenvironment{definition}[1][]{
  \begin{flushleft}
  \begin{addmargin}[2em]{0em}
  \text{Definition \refstepcounter{definitionno}\arabic{definitionno}}
  \textbf{#1}\\
  \vpush
  \begin{addmargin}[2em]{0em}
}{
  \end{addmargin}
  \end{addmargin}
  \end{flushleft}
}

\newcounter{sectiondefno}
\newenvironment{sectiondef}[1]{
  \begin{flushleft}
  \begin{addmargin}[2em]{0em}
  \textbf{Definition \thesection.\refstepcounter{sectiondefno}\arabic{sectiondefno}}
  \textbf{#1}\\
  \vpush
  \begin{addmargin}[2em]{0em}
}{
  \end{addmargin}
  \end{addmargin}
  \end{flushleft}
}

\makeatletter
\newcommand*{\inlineequation}[2][]{%
  \begingroup
    % Put \refstepcounter at the beginning, because
    % package `hyperref' sets the anchor here.
    \refstepcounter{equation}%
    \ifx\\#1\\%
    \else
      \label{#1}%
    \fi
    % prevent line breaks inside equation
    \relpenalty=10000 %
    \binoppenalty=10000 %
    \ensuremath{%
      % \displaystyle % larger fractions, ...
      #2%
    }%
    ~\@eqnnum
  \endgroup
}
\makeatother

\setlength{\fboxsep}{12pt}


\title{Convex Optimization 725 Lecture 3: Convexity II: Optimization Basics}
\date{2021-02-22}
\author{John Colvin}

\begin{document}
\maketitle

\begin{theorem}[First-Order Characterization]
  If $f$ is differentiable, then $f$ is convex if and only if $dom(f)$ is convex, and
  \begin{align}
    f(y)\geq f(x)+\nabla f(x)^T(y-x):\{\forall x,y\in dom(f)\}
  \end{align}
\end{theorem}
\begin{theorem}[Second-Order Characterization]
  If $f$ is twice differentiable, $f$ is convex if and only if $dom(f)$ is convex, and
  \begin{align}
    \nabla^2f(x)\succeq 0:\{\forall x\in dom(f)\}
  \end{align}
\end{theorem}
\begin{definition}[Affine Composition]
  \begin{align}
    \stackrel{convex}{f}\implies \stackrel{convex}{g(x)=f(Ax+b)}
  \end{align}
\end{definition}
\begin{definition}[General Composition]
  Suppose
  \begin{align}
    f=h\circ g,~g:\mathbb{R}^n\rightarrow \mathbb{R}
  \end{align}
  Then\\
  \vpush
  \begin{itemize}
    \item $f$ is convex if $h$ is convex and nondecreasing, $g$ is convex
    \item $f$ is convex if $h$ is convex and nonincreasing, $g$ is concave
    \item $f$ is concave if $h$ is concave and nondecreasing, $g$ is concave
    \item $f$ is concave if $h$ is concave and nonincreasing, $g$ is convex
  \end{itemize}
  Think chain rule:
  \begin{align}
    f^{\prime\prime}(x)=h^{\prime\prime}(g(x))g^\prime(x)^2+h^\prime(g(x))g^{\prime\prime}(x)
  \end{align}
  % \begin{tabular}{ c c c }
  %   $f$    & $g$                      & $h$\\
  %   \cmidrule{1-3}
  %   convex if: & convex and nondecreasing & convex\\
  %   convex if: & convex and nonincreasing & concave\\
  %   concave if: & concave and nondecreasing & concave\\
  %   concave if: & concave and nonincreasing & convex
  % \end{tabular}
\end{definition}
\begin{headered_note}[Example: Log-Sum-Exp Function]
  Proove the log-sum-exp function $g(x)$ is convex.
  \begin{align}
    g(x)=\log \left(\sum_{i=i}^ke^{a_i^Tx+b_i}\right)
  \end{align}
  By the \textbf{Affine Composition Rule},
  \begin{align}
    \stackrel{convex}{f(x)} & \implies \stackrel{convex}{g(x)=f(Ax+b)} \\
  \end{align}
  Replace the $Ax+b$ with $x$ in $f(x)$ to make the proof simpler (the
  simplified $f(x)$'s convexity implies $f(Ax+b)$ convexity)
  \begin{align}
    f(x) & =\log \left(\sum_{i=1}^k e^{x_i}\right)
  \end{align}
  If $f(x)$ is convex, $g(x)$ must be convex.  Use the \textbf{Second Order Characterization}:
  \begin{align}
    \nabla_if(x)={e^{x_i}\over \sum_{\ell=1}^n e^{x_i}} \\
    \nabla_i^2f(x)={e^{x_i}\over \sum_{\ell=1}^n e^{x_i}}\{i=j\}-{e^{x_i}e^{x_j}\over \left(\sum_{\ell=1}^ne^{x_i}\right)^2}
  \end{align}
  Abstract out the sum to simplify the expression and easily demonstrate convexity
  \begin{align}
    z_i          & ={e^{x_i}\over \left(\sum_{\ell=1}^ne^{x_i}\right)} \\
    \nabla^2f(x) & =diag(z)-zz^T
  \end{align}
  Since $z$ is diagonally dominant, it is \textbf{Positive Semidefinite}, which means
  \begin{align}
    \nabla^2f(x)\geq 0\implies \stackrel{convex}{f(x)}\implies \stackrel{convex}{g(x)}
  \end{align}
\end{headered_note}
\begin{headered_note}[Example: Support Vector Machine]
  \begin{align}
     & \min_{\beta,\beta_0,\xi} & {1\over 2}||\beta||_2^2+C\sum_{i=1}^n\xi_i     \\
    \nonumber
     & \text{Subject to }       & \xi_i\geq 0,i=1,...,n                          \\
    \nonumber
     &                          & y_i(x_i^T\beta +\beta_0)\geq 1-\xi_i,i=1,...,n
  \end{align}
\end{headered_note}
\begin{headered_note}[Rewriting Constraints]
  All represent the same optimization problem.
  \begin{align*}
     & 1. & \min_x             & f(x)                   \\
     &    & \text{subject to } & g_i(x)\leq 0,i=1,...,m \\
     & 2. &                    & Ax=b                   \\
     & 3. & \min_x             & f(x)+I_C(x)
  \end{align*}
  Where $I_C(x)$ is the \textbf{indicator function} for $x\in dom f$.
\end{headered_note}
\begin{theorem}[First Order Optimality Condition]
  For a \textbf{convex problem} \refeq{eq:firstordopt}, and a
  \textbf{differentiable} $f$, a feasible point $x$
  is optimal if and only if \refeq{eq:firstordoptdiff} is true.
  \begin{align}
    \label{eq:firstordopt}
    &\min_x f(x)&x\in C\\
    \label{eq:firstordoptdiff}
    &\nabla f(x)^T(y-x)\geq 0&\forall y\in C
  \end{align}
  \begin{headered_note}[Implications]
    All feasible directions from $x$ are aligned with gradient $\nabla f(x)$ 
  \end{headered_note}
  \begin{headered_note}[Important special case:]
    If $C=\mathbb{R}^n$ (unconstrained optimization), then optimality condition
    reduces to familiar $\nabla f(x)=0$
  \end{headered_note}
\end{theorem}
\begin{headered_note}[Example: Quadratic Minimization]
  Consider minimizing the quadratic function
  \begin{align}
    f(x)={1\over2}x^TQx+b^Tx+c
  \end{align}
  where $Q\succeq0$.  The first-order condition says that solution satisfies
  \begin{align}
    \nabla f(x)=Qx+b=0
  \end{align}
  \begin{headered_note}[Cases:]
    \begin{itemize}
      \item if $Q\succ0$ then there is a unique solution $x=Q^{-1}b$
      \item if $Q$ is singular and $b\notin col(Q)$ then there is no solution
        (i.e., $\min_xf(x)=-\infty)$
      \item if $Q$ is singular and $b\in col(Q)$ then there are infinitely many solutions
        \begin{align}
          x=Q^+b+x,~~x\in \text{null}(Q)
        \end{align}
    \end{itemize}
  \end{headered_note}
\end{headered_note}
\begin{headered_note}[Example: Equality Constrained Minimization]
  Consider the equality constrained convex problem:
  \begin{align}
    \min_x f(x) \text{ subject to }Ax=b
  \end{align}
  with differentiable $f$.  Prove the \textbf{Lagrange Multiplier} optimality condition
  \begin{align}
    \nabla f(x)+A^Tu=0\text{ for some }u
  \end{align}
  According to \textbf{First-Order Optimality}, solution $x$ satisfies $Ax=b$ and
  \begin{align}
    \nabla f(x)^T(y-x)\geq 0~~\{\forall y | Ay=b\}
  \end{align}
\end{headered_note}
\begin{headered_note}[Transformations and change of variables]
  If h: $\mathbb{R}\rightarrow \mathbb{R}$ is a \textbf{monotone increasing transformation}, then
  \begin{align}
    \min_x f(x)\text{ subject to }x\in C\\
    \Longleftrightarrow \min_x h(f(x))\text{ subject to }x\in C
  \end{align}
  Similarly, inequality or equality constraints can be transformed and yield
  equivalent optimization problems.  Can use this to reveal \textquoteleft hidden convexity\textquoteright of a problem.
  \\
  \vpush
  If $\phi:\mathbb{R}^n\rightarrow \mathbb{R}^m$ is one-to-one, and its image convers feasible set $C$,
  then we can change variables in an optimization problem:
  \begin{align}
    \min_x f(x)\text{ subject to }x\in C\\
    \Longleftrightarrow min_x f(\phi (y))\text{ subject to }\phi(y)\in C
  \end{align}
\end{headered_note}
\end{document}
