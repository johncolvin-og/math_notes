\documentclass{article}
\usepackage{array}
\usepackage{amsmath}
\usepackage{amssymb}
\usepackage{mathtools}
\usepackage{stackrel}
\usepackage[dvipsnames,definecolor]{xcolor}
\usepackage{array}
\usepackage{amsmath}
\usepackage{amssymb}
\usepackage{mathtools}
\usepackage[T1]{fontenc}
\usepackage{stackrel}
\usepackage[colorlinks]{hyperref}
\usepackage[user,titleref]{zref}
\usepackage[dvipsnames,definecolor]{xcolor}
\usepackage[most]{tcolorbox}
\usepackage{scrextend}

\usepackage{enumitem}
\usepackage{framed}
\usepackage{multicol}

\usepackage{xcolor}
\colorlet{linkequation}{red}
\colorlet{linktheo}{green}
\usepackage[colorlinks]{hyperref}

\definecolor{propbg}{RGB}{180,220,255}
\definecolor{factbg}{RGB}{170,232,145}
\definecolor{formula_header}{RGB}{30,40,210}

\newcounter{theoremc}
\newcommand{\theoremc}{\refstepcounter{theoremc}{\noindent\textbf{Theorem} \arabic{theoremc}}}
\newcommand{\myparbox}{\parbox[p]{12cm}}
\newcommand{\vpush}{\vspace{3mm}}
\newcommand{\sectionbreak}{\makebox[\linewidth]{\rule{12cm}{0.4pt}}}
\newcommand{\minmax}[2]{\min\limits_{#1}~\max\limits_{#2}}
\newcommand{\maxmin}[2]{\max\limits_{#1}~\min\limits_{#2}}

\newenvironment{note}[1][]{
  \textbf{#1}
  \begin{flushleft}
  \begin{addmargin}[2em]{0em}
}{
  \end{addmargin}
  \end{flushleft}
}
\newenvironment{proof}
{
\noindent
\begin{addmargin}[2em]{0em}
\textbf{Proof}
}
{
\end{addmargin}
}

\newenvironment{headered_note}[1][]{
\begin{flushleft}
\begin{addmargin}[2em]{0em}
\textbf{#1\\}
\vpush
\begin{addmargin}[2em]{0em}
}
{
\end{addmargin}
\end{addmargin}
\end{flushleft}
}

\newenvironment{fact}[1][]{
\begin{flushleft}
\begin{addmargin}[2em]{0em}
\textbf{Fact: #1\\}
\vpush
\begin{addmargin}[2em]{0em}
}
{
\end{addmargin}
\end{addmargin}
\end{flushleft}
}

\newenvironment{theorem}[1][]{
\begin{tcolorbox}[colback=green!30,%gray background
colframe=black,% black frame colour
% 5cm,% Use 5cm total width,
arc=3mm, auto outer arc,
]
\theoremc: \space \textbf{#1}
\begin{addmargin}[2em]{0em}
}
{
\end{addmargin}
\end{tcolorbox}}

\newenvironment{formula}[1]{
\begin{tcolorbox}[colback=white,%gray background
colframe=black,% black frame colour
% 5cm,% Use 5cm total width,
arc=3mm, auto outer arc,
]
\begin{flushleft}
{\color{formula_header} \textbf{Formula}}
\textbf{#1}
\end{flushleft}
\begin{addmargin}[2em]{0em}
}
{
\end{addmargin}
\end{tcolorbox}}

\newenvironment{headered_box}[1]{
\begin{tcolorbox}[colback=white,%gray background
colframe=black,% black frame colour
% 5cm,% Use 5cm total width,
arc=3mm, auto outer arc,
]
\begin{flushleft}
\textbf{#1}
\end{flushleft}
\begin{addmargin}[2em]{0em}
}
{
\end{addmargin}
\end{tcolorbox}}

\newenvironment{red_rounded_box}{
\begin{tcolorbox}[enhanced jigsaw,
    colback=gray!30,%gray background
    colframe=Maroon,% black frame colour
    width=5cm,% Use 5cm total width,
    arc=3mm, auto outer arc,
    boxrule=5pt,
    drop shadow={Maroon!50!gray!80}
  ]
}
{ \end{tcolorbox} }

\newcommand*{\reftheo}[1]{%
  \begingroup
    \hypersetup{
      linkcolor=linktheo,
      linkbordercolor=linktheo,
    }%
    (\ref{#1})%
  \endgroup
}

\newcommand*{\refequation}[1]{%
  \begingroup
    \hypersetup{
      linkcolor=linkequation,
      linkbordercolor=linkequation,
    }%
    \ref{#1}%
  \endgroup
}

\newenvironment{axiom}{
  \begin{tcolorbox}[enhanced jigsaw,
    colback={blue!20},
    colframe=Maroon,
    arc=2mm, auto outer arc,
    boxrule=3pt,
    drop shadow={blue!60!gray!60}
  ]
}{
  \end{tcolorbox}
}
\newenvironment{proposition}[2]{
\noindent\fcolorbox{white}{propbg}{%
  \parbox{\textwidth}{%
    \textbf{Proposition #1}
    {#2}
  }
}
}

\newcounter{definitionno}
\newenvironment{definition}[1][]{
  \begin{flushleft}
  \begin{addmargin}[2em]{0em}
  \text{Definition \refstepcounter{definitionno}\arabic{definitionno}}
  \textbf{#1}\\
  \vpush
  \begin{addmargin}[2em]{0em}
}{
  \end{addmargin}
  \end{addmargin}
  \end{flushleft}
}

\newcounter{sectiondefno}
\newenvironment{sectiondef}[1]{
  \begin{flushleft}
  \begin{addmargin}[2em]{0em}
  \textbf{Definition \thesection.\refstepcounter{sectiondefno}\arabic{sectiondefno}}
  \textbf{#1}\\
  \vpush
  \begin{addmargin}[2em]{0em}
}{
  \end{addmargin}
  \end{addmargin}
  \end{flushleft}
}

\makeatletter
\newcommand*{\inlineequation}[2][]{%
  \begingroup
    % Put \refstepcounter at the beginning, because
    % package `hyperref' sets the anchor here.
    \refstepcounter{equation}%
    \ifx\\#1\\%
    \else
      \label{#1}%
    \fi
    % prevent line breaks inside equation
    \relpenalty=10000 %
    \binoppenalty=10000 %
    \ensuremath{%
      % \displaystyle % larger fractions, ...
      #2%
    }%
    ~\@eqnnum
  \endgroup
}
\makeatother

\setlength{\fboxsep}{12pt}

\title{Convex Optimization 725 Lecture 1: Introduction}
\date{2021-02-21}
\author{John Colvin}

\begin{document}
\maketitle
\section{Basic Definitions}
\begin{definition}[Convex Set]
  A subset of an \textit{affine space} over $\mathbb{R}$ in which the line
  segment connecting any two points in the set is itself contained in the set .
\end{definition}
\begin{definition}[Convex Function]
  Any function whose domain is a \textbf{convex set}.  This is expressed
  in mathematical terms as \textbf{Jensen's Inequality}:
  \begin{align}
    f(tx_1+(1-t)x_2)\leq tf(x_1)+(1-t)f(x_2)~:~\{0\leq t\leq 1\}
  \end{align}
  This says that, given any two values in the domain $x_1,x_2$ and
  their corresponding functions $y_1,y_2$, any particular weighted average
  between $x_1$ and $x_2$ is less than or equal to that same weighted average
  of $y_1$ and $y_2$.\\
  \vpush
  \textbf{My Thoughts:}
  \begin{itemize}
    \item annoyingly absent from this definition is the assumption that the
          function is NON-NEGATIVE
    \item What would happen if you took a parabola (convex) and shifted it
          vertically down so that it's minimum was negative?  Jensen's
          inequality wouldn't hold true!
    \item It seems like generally speaking ${d^2f(x)\over dx^2}>0~:~\forall x$.
    \item Every \textbf{real valued linear transformation} is convex, since by
          definition, \textit{linear} functions are distributive:
    \begin{align}
      f(a+b)=f(a)+f(b)\leq f(a)+f(b)
    \end{align}
          However, this is not \textbf{strictly convex} (see lecture 2 notes).
  \end{itemize}
  \textbf{Epiphany:}
  \begin{itemize}
    \item All this says, is that the chord from $\{x_1, f(x_1)\}$ to $\{x_2, f(x_2)\}$
    will always be entirely \textbf{above the curve} $f(x)$.  The chord represents
    all \textit{weighted averages} of $f(x_1)$ and $f(x_2)$, and the curve represents
    the solution path from $f(x_1)$ to $f(x_2)$.  The chord is always greater than or equal
    to the solution path $\forall x \in dom f$.
  \end{itemize}
  The domain of a \textbf{convex function} is an \textbf{intersection of convex sets}.
\end{definition}
\begin{note}
  Convex problems only have one minima (i.e., a \textit{minimum}).
  In other words, there are no local minima, only the global.\newline
  \\
  It follows that given a convex function $f(x)$, if it can be proven that
  $f(x):x\in D$ is a minimum in ONE direction, it must be the minimum in all
  directions (global minimum).\\
\end{note}
\section{Convex Optimization Problems}
\begin{headered_box}{Optimization Problem}
  \begin{tabular}{ c c }
    $\stackrel[x\in D]{\min}{}$ & $f(x)$                     \\
    % \cmidrule{1-2}
    subject to                  & $g_i(x)\leq 0,~~i=1,...,m$ \\
                                & $h_j(x)=0,~~j=1,...,m$
  \end{tabular}\\
  \begin{align}
    D      & \triangleq dom(f)\bigcap\left(\bigcap_{i=1}^m dom(g_i)\right)\bigcap \left(\bigcap_{j=1}^p dom(h_j)\right) \\
    h_j(x) & =a^Tx+b_j,~~j=1,...,p
  \end{align}\\
  \begin{itemize}
    \item $D$ is the common domain of all functions
    \item $f$ is convex
    \item $g_i,i=1,...,m$ is convex
    \item $h_i,i=1,...,m$ is affine
  \end{itemize}
\end{headered_box}
\begin{flushleft}
  \textbf{Random Notes}\\
  \begin{itemize}
    \item $f(x)=Log(x)$ is \textbf{concave}
    \item $f(x)=-Log(x)$ is \textbf{convex}
  \end{itemize}
\end{flushleft}
\end{document}
