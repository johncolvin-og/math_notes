\documentclass{article}
\usepackage{amsmath}
\usepackage{amsthm}
\usepackage{amsfonts}
\usepackage{cancel}
\usepackage{changepage}
\usepackage{xcolor}
\usepackage[user,titleref]{zref}

\usepackage{color}
\usepackage{xcolor}
\usepackage{listings}
\usepackage{showexpl}
\usepackage{xparse}
\usepackage{caption}
\usepackage{changepage}
\usepackage{kantlipsum}

\makeatletter
\newlength\dottedlinefillheight
\setlength\dottedlinefillheight{.25in}

\def\fillwithdottedlines{%
  \begingroup
  \ifhmode
    \par
  \fi
  \hrule height \z@
  \nobreak
  \setbox0=\hbox to \hsize{\hskip \@totalleftmargin
          \vrule height \dottedlinefillheight depth \z@ width \z@
          \dotfill}%
  \cleaders \copy0 \vfill \hbox{}%
  \endgroup
}
\makeatother

\title{Probability 18600}
\date{2020-11-08}
\renewcommand{\familydefault}{\sfdefault}
\theoremstyle{definition}
\newtheorem*{definition}{Definition}

\theoremstyle{axiom}
\newtheorem{axiom}{Axiom}

\theoremstyle{proposition}
\newtheorem{proposition}{Proposition}

\newcommand\numberthis{\addtocounter{equation}{1}\tag{\theequation}}

\NewDocumentEnvironment{sidebyside}{O{.50} o +m +m}{%
  \noindent\begin{minipage}[t][][t]{#1\linewidth}%
  #3% Content of the first minipage
  \end{minipage}%
  \hfill%
  \noindent\begin{minipage}[t][][t]{\IfValueTF{#2}{#2}{#1}\linewidth}%
  #4% Content of the second minipage
  \end{minipage}\\% newline is important, it allows \hfill to work correctly, try removing it ;)
}

\newcommand{\exampleIndent}{20pt}
\newcommand{\solutionIndent}{20pt}

\newcommand{\myparagraph}[1]{\paragraph{#1}\mbox{}\\}
\newcommand{\header}[1]{
  \begin{flushleft}
    \large
    \textbf{#1}
  \end{flushleft}
}

\newcommand{\lnbk}{\mbox{\\ \\ \\}}

\newcommand{\exref}[1]{ (ex \textbf{#1}) }

\newenvironment{propositionenv}[2]{
  \begin{proposition}
    \textbf{#1}
    \begin{adjustwidth}{\exampleIndent}{2pt} {
      {#2}
    }
    \end{adjustwidth}
  \end{proposition}
}

\newenvironment{example}[2]{
  \header{Example #1}
  \begin{adjustwidth}{\exampleIndent}{2pt} {
    {#2}
  }
  \end{adjustwidth}
}

\newenvironment{exampleOrig}[2]{
  \noindent\fbox{
    \parbox{\textwidth}{
  \large
  \textbf{Example #1}
  \normalsize
  {#2}
  }
  }
}

\newenvironment{textbox}
{\begin{center}
  \begin{tabular}{|p{0.9\textwidth}|}
    \hline \\
    }
    {
    \\\\\hline
  \end{tabular}
\end{center}
}

\newenvironment{solution}[1] {
\begin{flushleft}
  \large
  \textbf{Solution}
\end{flushleft}
\begin{adjustwidth}{\solutionIndent}{2pt} {
    {#1}
  }
\end{adjustwidth}
}

% \newenvironment{proposition}[1] {
%   \begin{flushleft}
%   \large
%   \textbf{Proposition #1}
%   \end{flushleft}
% }
\definecolor{propbg}{RGB}{180,220,255}
\newenvironment{propositionEnv}[2]{
\noindent\fcolorbox{white}{propbg}{%
  \parbox{\textwidth}{%
    \textbf{Proposition #1}
    {#2}
  }
}
}

\begin{document}
\maketitle
\section{Introduction}
Introducing \textbf{Conditional Probability}
\section{Conditional Probabilities}

\begin{definition}
  Given two events, $E$ and $F$, the probability that $E$ will occur if $F$ \text{has already} occurred
  is denoted
  \begin{align}
    P(E|F)=\frac{P(EF)}{P(F)}~\{P(F)>0\}
  \end{align}
  For example, suppose 2 dice are rolled, $d_{1}$ and $d_{2}$.  Suppose $E$ is the event where $d_{1}+d_{2}=8$,
  and $F$ is the event where $d_{1}=3$.
  \begin{align*}
    P(F)   & =\frac{1}{6}                                  \\
    P(EF)  & =\frac{1}{36}                                 \\
    P(E|F) & =\frac{\frac{1}{36}}{\frac{1}{6}}=\frac{1}{6}
  \end{align*}
  I originally worked a different problem out, and \textit{just discovered} that
  it is NOT involved in the actual question in the book.
  Though I thought it was nice work, so I'm going to include it.
  \mbox{}\\ \\
  The odds that one particular number between 1 and 6 will appear on
  \textit{exactly} one of $n$ dice rolls is:
  \begin{align*}
    {n \choose 1}\frac{5^{n-1}}{6^n}
  \end{align*}
  If one die has the particular number, the rest \textit{must have a different number}.
  That leaves each of the remaining $n-1$ die with 5 possible outcomes, hence $5^{n-1}$.
  And there are $n$ different dice on which the particular number could be rolled, hence
  the $n\choose1$.  The last term, $6^n$ represents the total number of possible outcomes.
  \mbox{}\\ \\
  It follows then, that the odds such a particular number will appear on \textit{exactly two}
  die rolls is:
  \begin{align*}
    {n\choose2}\frac{5^{n-2}}{6^n}
  \end{align*}
  In general, the odds that a particular number will appear on $m~\{~1\leq m\leq n\}$ die rolls is:
  \begin{align*}
    \frac{1}{6^n}\sum_{i=1}^{n}{n\choose i}5^{n-i}
  \end{align*}
  Plug in $n=2,~m=2$ to get:
  \begin{align*}
    P(F)=\frac{1}{6^2}\left[{2\choose1}5^1+{2\choose2}5^0\right]=\frac{11}{36}
  \end{align*}
\end{definition}
\begin{example}{2a}{
    Joe is 80 percent certain that his missing key is in one of the two pockets of his
    hanging jacket, being 40 percent certain it is in the left-hand pocket and
    percent certain it is in the right-hand pocket. If a search of the left-hand pocket
    does not find the key, what is the conditional probability that it is in the other
    pocket?
  }
\end{example}
\begin{solution}{
    \begin{align*}
      P(R|L^{c}) & =\frac{P(RL^{c})}{P(R)}=\frac{.40}{.60}=\frac{2}{3}
    \end{align*}
  }
\end{solution}
\begin{example}{2b}{
    A coin is flipped twice. Assuming that all four points in the sample space
    $S=\{(h,h),(h,t),(t,h),(t,t)\}$ are equally likely, what is the conditional probability
    that both flips land on heads, given that:
    \begin{enumerate}
      \item the first flip lands on heads?
            \begin{align*}
              1\over2
            \end{align*}
      \item at least one flip lands on heads?
            \begin{align*}
              1\over3
            \end{align*}
    \end{enumerate}
  }
\end{example}
\begin{example}{2c} {
    In the card game bridge, the 52 cards are dealt out equally to 4 players–called
    East, West, North, and South. If North and South have a total of 8 spades among
    them, what is the probability that East has 3 of the remaining 5 spades?
  }
\end{example}
\begin{solution} {
    There are 26 cards left to deal evenly to \textit{East} and \textit{West}, of which 5 are spades.
    The probability that \textit{West} will receive 3 spades is:
    \begin{align*}
      {5\choose3}{21\choose10}\over {26\choose13}
    \end{align*}
  }
\end{solution}

\begin{example}{2d} {
    Celine is undecided as to whether to take a French course or a chemistry course.
    She estimates that her probability of receiving an A grade would be $1\over2$ in a French
    course and $2\over3$ in a chemistry course. If Celine decides to base her decision on the
    flip of a fair coin, what is the probability that she gets an A in chemistry?
  }
\end{example}
\begin{solution} {
    There are 4 possible outcomes: Celine 1) gets an A in chemistry, or 2) gets an A in french,
    or 3) doesn't get an A in chemistry, or 4) doesn't get an A in french.
    \begin{align*}
      E_1 & =Chemsitry~A  & P(E_{1}) & =\frac{1}{2}\cdot\frac{2}{3} \\
      E_2 & =Chemistry~<A & P(E_{2}) & =\frac{1}{2}\cdot\frac{1}{2} \\
      E_3 & =French~A     & P(E_{3}) & =\frac{1}{2}\cdot\frac{1}{3} \\
      E_4 & =French~<A    & P(E_{4}) & =\frac{1}{2}\cdot\frac{1}{2}
    \end{align*}
  }
\end{solution}
\begin{example}{2e}{
    Suppose that an urn contains 8 red balls and 4 white balls. We draw 2 balls from
    the urn without replacement.
    \begin{enumerate}
      \item If we assume that at each draw, each ball in the
            urn is equally likely to be chosen, what is the probability that both balls drawn are
            red?
            \begin{solution} {
                \begin{align*}
                  \frac{8}{12}\frac{7}{11}=\frac{56}{132}=\frac{14}{33}
                \end{align*}
                Note that this could also be expressed:
                \begin{align*}
                  P(R_1R_2)={{8\choose2}\over{12\choose2}}
                \end{align*}
              }
            \end{solution}
      \item Now suppose that the balls have different weights, with each red ball
            having weight $r$ and each white ball having weight $w$. Suppose that the
            probability that a given ball in the urn is the next one selected is its weight divided
            by the sum of the weights of all balls currently in the urn. Now what is the
            probability that both balls are red?
            \begin{align*}
              \frac{8r}{8r+4w}\frac{7r}{7r+4w}=\frac{56r^2}{56r^2+60wr+16w^2}=\frac{14^2}{14^2+15+4w^2}
            \end{align*}
    \end{enumerate}
  }
\end{example}
\begin{definition}{
    The Multiplication Rule states that $P(E_1...E_n)$, the probability that all of the events
    $E_1...E_n$ occur, is equal to $P(E_1)\cdot P(E_2|E_1)\cdot P(E_3|E_1E_2)\cdot...\cdot P(E_n|E_1...E_{n-1})$
  }
\end{definition}
\begin{example}{2f}{
    In the match problem stated in Example 5m of Chapter 2 , it was shown
    that $P_N$, the probability that there are no matches when $N$ people randomly select
    from among their own $N$ hats, is given by
    \begin{align}
      \label{eq:nomatchinghats}
      P_N=\sum_{i=0}^{N}\frac{(-1)^i}{i!}
    \end{align}
  }
\end{example}
\begin{solution} {
    This can be thought of as 2 smaller match problems:
    \begin{align*}
      P[(E_1,...,E_k)E_{k+1}^c,...,E_n^c]=
      P(E_1,...,E_k)P(E_{k+1}^c,...,E_n^c)
    \end{align*}
    The formulas for both probabilities in the product have already been established (in 5m).
    The probability that $n\in N$ men select their own hats is:
    \begin{align*}
      P(E_{i_1}...E_{i_n})=\frac{(N-n)!}{N!}={N\choose n}
    \end{align*}
    And the probability that 0 out of $k$ men choose their hat is
    \begin{align}
      P_{N-k}=\sum_{i=0}^{N-k}\frac{(-1)^i}{i!}
    \end{align}
    Thus the answer is
    \begin{align*}
      {N\choose k}P_{N-k}={N\choose k}\sum_{i=0}^{N-k}\frac{(-1)^i}{i!}
    \end{align*}
  }
\end{solution}
\begin{example}{2g} {
    An ordinary deck of 52 playing cards is randomly divided into 4 piles of 13 cards
    each. Compute the probability that each pile has exactly 1 ace.
  }
\end{example}
\begin{solution}{
    This makes sense to me, but it doesn't appear to be right (though I haven't verified this yet)
    \begin{align*}
      {4!{48\choose12}{36\choose12}{24\choose12}\over52!} & =4!\frac{48!}{(36!)(12!)}\frac{36!}{(24!)(12!)}\frac{24!}{(12!)^2}
                                                          & =4!
    \end{align*}
    From the book: consider these events (where $A_1,..,A_4$ are the aces of spades, hearts,
    diamonds, and clubs, respectively).
    \begin{align*}
      E_1 & =A_1~in~any~pile               \\
      E_2 & =A_1,A_2~in~diff~piles         \\
      E_3 & =A_1,A_2,A_3~in~diff~piles     \\
      E_4 & =A_1,A_2,A_3,A_4~in~diff~piles
    \end{align*}
    By the multiplication rule
    \begin{align*}
      P(E_1E_2E_3E_4)=P(E_1)P(E_2|E_1)P(E_3|E_1E_2)P(E_4|E_1E_2E_3)
    \end{align*}
    Of course, $P(E_1)=1$, so $A_1$ can simply be assumed to be one deck, then with
    51 cards remaining, that deck has a $12\over51$ chance of drawing $A_2$.  Therefore
    \begin{align*}
      P(E_2|E_1)=1-{12\over51}={39\over51}
    \end{align*}
    Presuming $E_1E_2$, $E_3$ will be true as long as the ace of diamonds isn't
    dealt to either the pile with the ace of spades, or the one with the ace of hearts.
    24 more cards must be dealt to those 2 decks, so the odds
    that neither pile receives the ace of diamonds is:
    \begin{align*}
      P(E_3|E_1E_2)=1-{24\over50}={26\over50}
    \end{align*}
    By similar logic
    \begin{align*}
      P(E_4|E_1E_2E_3)=1-{36\over39}={13\over49}
      \implies
      P(E_1E_2E_3E_4)={39\cdot26\cdot13\over51\cdot50\cdot49}
    \end{align*}
  }
\end{solution}
\begin{example}{2h}{
    Four of the eight teams in the quarterfinal round of the 2016 European
    Champions League Football (soccer) tournament were the acknowledged strong
    teams Barcelona, Bayern Munich, Real Madrid, and Paris St-Germain. The
    pairings in this round are supposed to be totally random, in the sense that all
    possible pairings are equally likely. Assuming this is so, find the probability that
    none of the strong teams play each other in this round. (Surprisingly, it seems to
    be a common occurrence in this tournament that, even though the pairings are
    supposedly random, the very strong teams are rarely matched against each
    other in this round.)
  }
\end{example}
\begin{solution} {
    This can be worked out in a similar way to \textbf{2g}.  Let $T_1,...,T_4$ be the teams
    Barcelona, Bayern Munich, Real Madrid, and Paris St-Germain, repsectively.
    \begin{align*}
      E_1 & =T_1~in~any~game               \\
      E_2 & =T_1,T_2~in~diff~games         \\
      E_3 & =T_1,T_2,T_3~in~diff~games     \\
      E_4 & =T_1,T_2,T_3,T_4~in~diff~games
    \end{align*}
    By the multiplication rule
    \begin{align*}
      P(E_1E_2E_3E_4)=P(E_1)P(E_2|E_1)P(E_3|E_1E_2)P(E_4|E_1E_2E_3)
    \end{align*}
    $P(E_1)=1$, so with 7 teams remaining, there is a one in seven chance that Barcelona will play
    Bayern Munich in the first round
    \begin{align*}
      P(E_2|E_1)=1-{1\over7}={6\over7}
    \end{align*}
    Next, compute
    \begin{align*}
      P(E_3|E_1E_2)=1-{2\over6}={2\over3}
    \end{align*}
    Finally, compute
    \begin{align*}
      P(E_4|E_1E_2E_3)=1-{3\over5}={2\over5}
    \end{align*}
    Therefore the probability that no strong teams face eachother in round one is
    \begin{align*}
      P(E_1E_2E_3E_4)={6\cdot2\cdot2\over7\cdot3\cdot5}={24\over105}={8\over35}
    \end{align*}
    Another way to view the problem, is to consider the games as 4 pairs of teams.
    If each of the 4 strong teams are in separate games, there are $4!$ ways
    to pair them with the 4 weaker teams.  As there are ${{8\choose2}{6\choose2}{4\choose2}{2\choose2}\over4!}$
    total possible pairings, the probability that none of the 4 strong teams face eachother in round
    ond is (again):
    \begin{align*}
      {(4!)(4!)\over{8\choose2}{6\choose2}{4\choose2}} & =(4!)(4!)\cdot{(2!)\cancel{(6!)}\over8!}
      \cdot{(2!)\cancel{(4!)}\over\cancel{6!}}\cdot{(2!)(2!)\over\cancel{4!}}                                                             \\
                                                       & =4!{2^4\over8\cdot7\cdot6\cdot5}                                                 \\
                                                       & =4!{2\cdot\cancel{2^3}\over\cancel{8}\cdot7\cdot6\cdot5}={48\over210}={8\over35}
    \end{align*}
    The handy dandy formula for counting the number of ways to choose $p$ groups of size $q$ from $n$
    elements was derived by yours truly.  Above, $p=4$, $q=2$, and $n=8$ (i.e., choose 4 pairs of teams from a pool of 8).
    It applies the $a~choose~b$ logic with a recursive step (albeit just one in this case, but it can be extended
    to count the number of ways to choose $c$ groups of $d$ groups of size $e$ from $f$ elements, and so on).
    \mbox{}\\ \\
    Consider the example at hand: there are $8\choose2$ ways to pick the first pair.  Then, for each of those choices, there are
    $6\choose2$ ways to pick the second pair, and so on.  The product ${8\choose2}{6\choose2}$ counts \textit{all permutations} of
    sets of 2 pairs.
    \mbox{}\\ \\
    For example, given 2 pairs $p_1$ and $p_2$, both permutations of those pairs ($p_1,p_2$ and $p_2,p_1$) are counted.
    In the scenario where $p_1$ is picked first, the 2 values that make up $p_2$ are part of the 6 items available for the second pair.
    Likewise, in the scenario where $p_2$ is picked first, the values of $p_1$ are available for the second pair.
    \mbox{}\\ \\
    Recall, ${a\choose b}={a!\over{(a-b)!(b!)}}$; the $b!$ is in the denominator to divide by the number of permutations
    per chosen value-set.  Notice how this formula similarly has a sort of factorial of \textit{choose} terms in the numerator,
    ${8\choose2}{6\choose2}{4\choose2}{2\choose2}$.
  }
\end{solution}

\section{Bayes's Formula}
\begin{align*}
  P(E) & =P(EF) + P(EF^c)                          \\
       & =P(E|F)P(F) + P(E|F^c)(1-P(F))\numberthis
\end{align*}
\mbox{}\\
\begin{example}{3a} {
    \begin{enumerate}
      \item
            An insurance company believes that people can be divided into two classes:
            those who are accident prone and those who are not. The company’s statistics
            show that an accident-prone person will have an accident at some time within a
            fixed 1-year period with probability .4, whereas this probability decreases to .2 for
            a person who is not accident prone. If we assume that 30 percent of the
            population is accident prone, what is the probability that a new policyholder will
            have an accident within a year of purchasing a policy?
            \begin{solution} {
                Let $B$ be the event where a \textit{bad} driver is picked (accident prone), and $G$ be the event
                where a \textit{good} driver is picked (not accident prone).  Let $A$ be the event where the chosen
                driver has an accident in the first year.
                \begin{align*}
                  P(A) & =P(A|B)P(B)+P(A|G)P(G) \\
                       & =(.4)(.3)+(.2)(.7)=.26
                \end{align*}
              }
            \end{solution}
      \item Suppose that a new policyholder has an accident within a year of purchasing a
      policy. What is the probability that he or she is accident prone?
      \begin{solution} {
        \begin{align*}
          P(B|A)&={P(BA)\over P(A)}\\
          &={.3\cdot.4\over.26}={6\over13}
        \end{align*}
      }
      \end{solution}
    \end{enumerate}
  }
\end{example}
\begin{example}{3b}{
  Consider the following game played with an ordinary deck of 52 playing cards:
The cards are shuffled and then turned over one at a time. At any time, the
player can guess that the next card to be turned over will be the ace of spades; if
it is, then the player wins. In addition, the player is said to win if the ace of
spades has not yet appeared when only one card remains and no guess has yet
been made. What is a good strategy? What is a bad strategy?
}
\end{example}
\begin{solution}{
  This is a trick question: the ace of spades is equally likely to be in any one of the 52 possible positions.
  This is akin to asking someone to pick a number between 1 and 52, if you get it right, you win (proximity
  is awarded nothing).
}
\end{solution}
\begin{example}{3c}{
  In answering a question on a multiple-choice test, a student either knows the
answer or guesses. Let $p$ be the probability that the student knows the answer
and $1-p$ be the probability that the student guesses. Assume that a student who
guesses at the answer will be correct with probability $1/m$, where $m$ is the
number of multiple-choice alternatives. What is the conditional probability that a
student knew the answer to a question given that he or she answered it
correctly?
}
\end{example}
\begin{solution}{
  \begin{align*}
    P(K|C)={P(C|K)P(K)\over P(C)}
    ={P(KC)\over P(C)}
    ={P(K)\over P(C)}
  \end{align*}
  and
  \begin{align*}
    P(C)&=P(C|K)P(K)+P(C|K^c)P(K^c)=p+{1\over m}\\
    \implies P(K|C)&=p({1\over p+{1\over m}(1-p)})
  \end{align*}
} 
\end{solution}
\begin{example}{3d} {
 A laboratory blood test is 95 percent effective in detecting a certain disease when
it is, in fact, present. However, the test also yields a “false positive” result for 1\%
of the healthy persons tested. If .5\% of the population actually has the disease,
what is the probability that a person has the disease given that the test result is positive?
}
\end{example}
\begin{solution}{
    \begin{align*}
      P(A)&=P(A|D)P(D)+P(A|D^c)P(D^c)\\
      P(A)&=.95(.005)+.01(.995)=.147\\
      P(D|A)&={P(AD)\over P(A)}\\
      P(D|A)&={.95(.005)\over .147}={95\over294}\approx.323
    \end{align*}
} 
\end{solution}
\end{document}