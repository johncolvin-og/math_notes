\documentclass{article}
\usepackage{amsmath}
\usepackage{changepage}
\usepackage{xcolor}
\usepackage[user,titleref]{zref}
\title{Probability 18600}
\date{2020-11-07}

\newcommand{\exampleIndent}{20pt}
\newcommand{\solutionIndent}{20pt}

\newcommand{\myparagraph}[1]{\paragraph{#1}\mbox{}\\}
\newcommand{\header}[1]{
  \begin{flushleft}
    \large
    \textbf{#1}
  \end{flushleft}
}

\newcommand{\lnbk}{\mbox{\\ \\ \\}}

\newcommand{\exref}[1]{ (ex \textbf{#1}) }

\newenvironment{example}[2]{
  \header{Example #1}
  \begin{adjustwidth}{\exampleIndent}{2pt} {
    {#2}
  }
  \end{adjustwidth}
}

\newenvironment{exampleOrig}[2]{
  \noindent\fbox{
    \parbox{\textwidth}{
  \large
  \textbf{Example #1}
  \normalsize
  {#2}
  }
  }
}

\newenvironment{textbox}
{\begin{center}
  \begin{tabular}{|p{0.9\textwidth}|}
    \hline \\
    }
    {
    \\\\\hline
  \end{tabular}
\end{center}
}

\newenvironment{solution}[1] {
\begin{flushleft}
  \large
  \textbf{Solution}
\end{flushleft}
\begin{adjustwidth}{\solutionIndent}{2pt} {
    {#1}
  }
\end{adjustwidth}
}

% \newenvironment{proposition}[1] {
%   \begin{flushleft}
%   \large
%   \textbf{Proposition #1}
%   \end{flushleft}
% }
\definecolor{propbg}{RGB}{180,220,255}
\newenvironment{proposition}[2]{
\noindent\fcolorbox{white}{propbg}{%
  \parbox{\textwidth}{%
    \textbf{Proposition #1}
    {#2}
  }
}
}

\begin{document}
\maketitle
\section{hello}\zlabel{sec:one}
The section name is: \ztitleref{sec:one}.
\section{1.6 The Number of Integer Solutions to Equations}
\paragraph{What is this?}
Given a set of n items, and r divisons of size $n_1,...,n_r$,
such that $\sum n_{i}=n$, the number of ways in which the items could be divided
into such groups is the \textit{Number of Integer Solutions}.

\paragraph{Why use it?}
A question that calls for such a solution could be the following:
A man goes fishing at Lake Toconderoga, which contains
four types of fish: lake trout, catfish, bass, and bluefish.  If he caught
10 fish total, how many combinations of the 4 fish could he have caught?

\paragraph{}
The outcome of the fishing trip could be represented as a vector, with one entry per fish type:
$\vec{x}=(x_{1},x_{2},x_{3},x_{4})$.
The the question is: how many positive integer valued vectors satisfy the following equation?
\begin{equation}
  \label{vec_entry_sum}
  x_{1}+x_{2}+x_{3}+x_{4}=10
\end{equation}
Suppose there are $n$ consecutive zeros lined up in a row:
\begin{equation*}
  000...000
\end{equation*}
Each arrangement of the $r-1$ into the $n-1$ spaces between adjacent zeros corresponds to a positive
solution to \eqref{vec_entry_sum}, where $x_{i}$ is the number of zeros between the $(i-1)^{th}$, and $i^{th}$ dots.
\\ \\
For instance, if $n=8$, and $r=3$, then the choice:
\begin{equation*}
  0.0000.000 \implies x_{1}=1,~x_{2}=4,~x_{3}=3
\end{equation*}
\\ \\
Therefore, there are $n-1 \choose r-1$ distinct \textit{Positive Integer Valued Vectors} $(x_{1},...,x_{r})$
that satisfy the equation:
\begin{equation*}
  x_{1}+...+x_{r}=n,~x_{i} > 0,~i=1,...,r
\end{equation*}
\paragraph{To count nonnegative} integer solutions, let $y_{i}=x_{i}+1$ and count the solutions to
\begin{equation*}
  y_{1}+...+y_{r}=n+r
\end{equation*}
\begin{proposition}{6.1}{
    The number of \textit{Nonnegative Integer Valued Vectors} satisfying \eqref{vec_entry_sum} is
    \begin{equation*}
      n+r-1 \choose r-1
    \end{equation*}
}
\end{proposition}
\mbox{}\\

\begin{example}{6c} {
    How many terms are there in the multinomial expansion of $(x_{1}+...+x_{r})^{n}$?
  }
\end{example}
\mbox{}\\
\begin{solution} {
    there are $n+r-1 \choose r-1$ terms in the polynomial, as each term is a unique division
    of the $n$ degrees of powers spread across the $r$ variables.
  }
\end{solution}
\mbox{}\\
\begin{example}{6d} {
    Consider a set of $n$ indistinguishable antennas, of which $m<n$ are defective,
    and which must be arranged in a sequence such that the defective antennas are separated by at least
    1 functional antenna.
    Imagine the defective items are lined up among themselves, and all that's left to do
    is insert the functional ones between them.  There are $m+1$ spaces in which the functional
    antenna may be inserted.  There are $m+1$ groups in which the functional antenna must be \textbf{divided}.
    All but the first and last group must have at least 1 antenna.
  }
\end{example}
\begin{solution} {
    \begin{align*}
       & x_{1}+...+x_{m+1}=n-m,          \\
       & x_{i}>0~\{i~|~2\leq i \leq m\}, \\
       & x_{1}\geq 0,~x_{m+1}\geq 0
    \end{align*}
    \paragraph{}So it follows that if
    \begin{align*}
       & y_{i}=x_{i}~\{i~|~2\leq i \leq m\} \\
       & y_{1}=x_{0}+1,~y_{m+1}=x_{m+1}+1
    \end{align*}
    \paragraph{}Then the number of possible antenna arrangements is the same as the number of
    positive integer solutions to the equation if \exref{6c} then
    \begin{align*}
      y_{1}+...+y_{m+1}=n-m+1 \implies arrangements={n-m+1 \choose m}
    \end{align*}
  }
\end{solution}
\end{document}