\documentclass{article}
\usepackage{array}
\usepackage{amsmath}
\usepackage{amssymb}
\usepackage{mathtools}
\usepackage{stackrel}
\usepackage{scrextend}
\usepackage{enumitem}
\usepackage{array}
\usepackage{amsmath}
\usepackage{amssymb}
\usepackage{mathtools}
\usepackage[T1]{fontenc}
\usepackage{stackrel}
\usepackage[colorlinks]{hyperref}
\usepackage[user,titleref]{zref}
\usepackage[dvipsnames,definecolor]{xcolor}
\usepackage[most]{tcolorbox}
\usepackage{scrextend}

\usepackage{enumitem}
\usepackage{framed}
\usepackage{multicol}

\usepackage{xcolor}
\colorlet{linkequation}{red}
\colorlet{linktheo}{green}
\usepackage[colorlinks]{hyperref}

\definecolor{propbg}{RGB}{180,220,255}
\definecolor{factbg}{RGB}{170,232,145}
\definecolor{formula_header}{RGB}{30,40,210}

\newcounter{theoremc}
\newcommand{\theoremc}{\refstepcounter{theoremc}{\noindent\textbf{Theorem} \arabic{theoremc}}}
\newcommand{\myparbox}{\parbox[p]{12cm}}
\newcommand{\vpush}{\vspace{3mm}}
\newcommand{\sectionbreak}{\makebox[\linewidth]{\rule{12cm}{0.4pt}}}
\newcommand{\minmax}[2]{\min\limits_{#1}~\max\limits_{#2}}
\newcommand{\maxmin}[2]{\max\limits_{#1}~\min\limits_{#2}}

\newenvironment{note}[1][]{
  \textbf{#1}
  \begin{flushleft}
  \begin{addmargin}[2em]{0em}
}{
  \end{addmargin}
  \end{flushleft}
}
\newenvironment{proof}
{
\noindent
\begin{addmargin}[2em]{0em}
\textbf{Proof}
}
{
\end{addmargin}
}

\newenvironment{headered_note}[1][]{
\begin{flushleft}
\begin{addmargin}[2em]{0em}
\textbf{#1\\}
\vpush
\begin{addmargin}[2em]{0em}
}
{
\end{addmargin}
\end{addmargin}
\end{flushleft}
}

\newenvironment{fact}[1][]{
\begin{flushleft}
\begin{addmargin}[2em]{0em}
\textbf{Fact: #1\\}
\vpush
\begin{addmargin}[2em]{0em}
}
{
\end{addmargin}
\end{addmargin}
\end{flushleft}
}

\newenvironment{theorem}[1][]{
\begin{tcolorbox}[colback=green!30,%gray background
colframe=black,% black frame colour
% 5cm,% Use 5cm total width,
arc=3mm, auto outer arc,
]
\theoremc: \space \textbf{#1}
\begin{addmargin}[2em]{0em}
}
{
\end{addmargin}
\end{tcolorbox}}

\newenvironment{formula}[1]{
\begin{tcolorbox}[colback=white,%gray background
colframe=black,% black frame colour
% 5cm,% Use 5cm total width,
arc=3mm, auto outer arc,
]
\begin{flushleft}
{\color{formula_header} \textbf{Formula}}
\textbf{#1}
\end{flushleft}
\begin{addmargin}[2em]{0em}
}
{
\end{addmargin}
\end{tcolorbox}}

\newenvironment{headered_box}[1]{
\begin{tcolorbox}[colback=white,%gray background
colframe=black,% black frame colour
% 5cm,% Use 5cm total width,
arc=3mm, auto outer arc,
]
\begin{flushleft}
\textbf{#1}
\end{flushleft}
\begin{addmargin}[2em]{0em}
}
{
\end{addmargin}
\end{tcolorbox}}

\newenvironment{red_rounded_box}{
\begin{tcolorbox}[enhanced jigsaw,
    colback=gray!30,%gray background
    colframe=Maroon,% black frame colour
    width=5cm,% Use 5cm total width,
    arc=3mm, auto outer arc,
    boxrule=5pt,
    drop shadow={Maroon!50!gray!80}
  ]
}
{ \end{tcolorbox} }

\newcommand*{\reftheo}[1]{%
  \begingroup
    \hypersetup{
      linkcolor=linktheo,
      linkbordercolor=linktheo,
    }%
    (\ref{#1})%
  \endgroup
}

\newcommand*{\refequation}[1]{%
  \begingroup
    \hypersetup{
      linkcolor=linkequation,
      linkbordercolor=linkequation,
    }%
    \ref{#1}%
  \endgroup
}

\newenvironment{axiom}{
  \begin{tcolorbox}[enhanced jigsaw,
    colback={blue!20},
    colframe=Maroon,
    arc=2mm, auto outer arc,
    boxrule=3pt,
    drop shadow={blue!60!gray!60}
  ]
}{
  \end{tcolorbox}
}
\newenvironment{proposition}[2]{
\noindent\fcolorbox{white}{propbg}{%
  \parbox{\textwidth}{%
    \textbf{Proposition #1}
    {#2}
  }
}
}

\newcounter{definitionno}
\newenvironment{definition}[1][]{
  \begin{flushleft}
  \begin{addmargin}[2em]{0em}
  \text{Definition \refstepcounter{definitionno}\arabic{definitionno}}
  \textbf{#1}\\
  \vpush
  \begin{addmargin}[2em]{0em}
}{
  \end{addmargin}
  \end{addmargin}
  \end{flushleft}
}

\newcounter{sectiondefno}
\newenvironment{sectiondef}[1]{
  \begin{flushleft}
  \begin{addmargin}[2em]{0em}
  \textbf{Definition \thesection.\refstepcounter{sectiondefno}\arabic{sectiondefno}}
  \textbf{#1}\\
  \vpush
  \begin{addmargin}[2em]{0em}
}{
  \end{addmargin}
  \end{addmargin}
  \end{flushleft}
}

\makeatletter
\newcommand*{\inlineequation}[2][]{%
  \begingroup
    % Put \refstepcounter at the beginning, because
    % package `hyperref' sets the anchor here.
    \refstepcounter{equation}%
    \ifx\\#1\\%
    \else
      \label{#1}%
    \fi
    % prevent line breaks inside equation
    \relpenalty=10000 %
    \binoppenalty=10000 %
    \ensuremath{%
      % \displaystyle % larger fractions, ...
      #2%
    }%
    ~\@eqnnum
  \endgroup
}
\makeatother

\setlength{\fboxsep}{12pt}


\newcommand\VRule[1][\arrayrulewidth]{\vrule width #1}
\title{Linear Algebra 18.065: Split Algorithms for $\ell^1$ Optimization}
\date{2021-02-17}
\author{John Colvin}

\begin{document}
\maketitle

\textbf{Sylvester Equation}
\begin{equation}
  AX-XB=C
\end{equation}
\vpush
\sectionbreak
\section{Split Algorithms for $\ell^2+\ell^1$}
When $Ax=b$ has many solutions (perhaps $A$ has more columns than rows,
$m<<n$), the \textit{optimal} solution is usually the one with the most zeros
(this generally reduces the number of arithmetic operations needed to apply the
solution in an algorithm).
\\ \\
The solution(s) with the most zeros minimize(s) the $\ell^0$ norm.  However,
since this isn't a \textit{real} norm (non-linear, non-distributive, etc.),
minimizing it isn't typically feasible.  A more feasible approach is to
minimize the $\ell^1$ norm.
\\ \\
The $\ell^1$ norm:
\begin{itemize}
  \item is a convex function of $x$
  \item is piecewise linear (unlike $\ell^2$, and
        all other $\ell^p$ norms except $||x||_\infty=\max{|x_i|}$)
  \item has vertices on the basis axes (the vectors with
        $||x_1||=|x_1|+|x_2|\leq 1$ fill the diamond with corners at $\pm i$ and $\pm j$
  \item solutions that just contact those vertices thus consist of basis vectors,
        which have only one non-zero element
\end{itemize}
\begin{formula}{Basis Pursuit}
  \textbf{Minimize}~~~~\inlineequation[eq:basispursuit]{
    ||x_1||=|x_1|+...+|x_n|~~\textbf{subject to }Ax=b~~~~~~~~~~~~~~~
  }
\end{formula}
\myparbox{
  A related problem allows for noise in $Ax=b$, as enforcing the constraint too strictly
  may lead to an \textit{overfitted} or suboptimal solution (e.g., too many non-zeros!).
  The \textbf{LASSO} approach addresses these issues.
}\\
\begin{formula}{LASSO(in statistics)}
  \begin{flushright}
    \begin{align}
                   & \textbf{Minimize}~~{1\over 2}||Ax-b||^2_2+\lambda||x_1||_1          \\
      (\text{or})~ & \textbf{Minimize}~~{1\over 2}||Ax-b||^2_2~\text{with}~||x||_1\leq t
    \end{align}
  \end{flushright}
\end{formula}
\vpush
\myparbox{
  Popular numerical implementations of \textbf{LASSO} are known as \textbf{Split Algorithms}.
  They \textit{split} the problem up into many smaller problems, which can be
  solved in parallel (independently), then merged back together into a final solution.
}
\\
\section{Alternating Direction Method of Multipliers (ADMM)}
ADDM begins by combining the cost function $f(x)$ with the constraint $Ax=b$.
\begin{headered_box}{Lagrangian}
  \inlineequation{L(x,y)=f(x)+y^T(Ax-b)=f(x)+y^T-y^Tb
    ~~~~~~~~~~~~~~~~~~~~~~~~}
\end{headered_box}
Then, the problem is decomposed into a separable differential equation.
\begin{headered_box}{Primal Problem}
  \inlineequation{\textbf{Minimize} f(x)\text{, subject to } Ax=b: A\text{ is } m \times n
    ~~~~~~~~~~~~~~~~~~~~}
\end{headered_box}
\begin{flushleft}
  The solution $x^\star$ depends on $y$.  By maximizing $y$, the constraint is enforced more strictly.
  \\
  \vspace{3mm}
  \textit{Note: there is a method to the madness of initializing y, but that will
    not be explored in this section.  For now, just assume that $y$ has been
    appropriately initialized.}
\end{flushleft}

\begin{headered_box}{Dual Problem}
  \inlineequation{\textbf{Maximize }m(y)=L(x^\star(y),y)
    ~~~~~~~~~~~~~~~~~~~~~~~~~~~~~~~~~~~~~~~~~~~~~~~~~~~~}
\end{headered_box}
To maximize $m(y)$, find a point $y^\star$ where all partial derivatives are zero.
\begin{align}
  \nabla m=\left({\partial m\over \partial y_1},...,{\partial m\over \partial y_m}\right)
\end{align}

\begin{center}
  \myparbox{
    \begin{center}
      \setlength{\fboxsep}{2pt}
      \fbox{
        \parbox[p]{11cm}{
          \begin{align}
            x_{k+1} & =\min L(x,y_k)       \\
            y_{k+1} & =y_k+s_k(Ax_{k+1}-b)
          \end{align}
          \begin{enumerate}
            \item Minimize $L$ to get $x_{k+1}$ (assume $y_{k}$ is known)
            \item Move $y$ [roughly] along the solution path by setting
                  the next $y_{k+1}=y_k+s_k(Ax_{k+1}-b)$.  This is a
                  'step' ($s_k$ is the step size) in the trajectory of the
                  maximum gradient (a sensible way to reach the maximium point
                  in the function).
          \end{enumerate}
        }
      }
    \end{center}
  }
\end{center}
The idea behind this algorithm is to find the \textit{best} solution to the
problem.  In this case, \textit{best} is defined to produce an optimal
combination of finding the solution to the original problem - some linear
system $Ax=b$ - and minimizing the $\ell_1$ norm!
\\ \\
It does this by iteratively
\textit{minimizing} $f(x)$ (i.e., the $\ell_1$ norm of $\hat{x}$), and \textit{maximizing}
$y$ (the penalty term coefficient in $y^T(A\hat{x}-b)$).
\\ \\
\section{Dual Decomposition}
Suppose the function to minimize $f(x)$ was \textbf{separable}
\begin{align*}
  f(x)=f_1(x_1)+...+f_n(x_n)
\end{align*}
Where $x_i$ is a \textit{subvector} of $x$ ($x$ is itself a vector, and
each $x_i$ is a different piece of $x$).  Together, the subvectors are $x$:
\begin{align}
  x=(x_1,...,x_n)
\end{align}
\\
Each of the $n$ individual functions $f(x_i)$ can be evaluated in parallel.
Every variable can be minized independently of the others (assuming $y$ is known).
There is a method to the madness of initializing $y$ with an appropriate value,
however that is not explored in this section.
\\ \\
The constraint matrix $A$ is broken up into chunks of $N$ columns to pair
with each $x_i$.  Finally, the Lagrangian is split up into $N$ pieces to be
computed in parallel:
\begin{align}
  \label{eq:nseparable}
  f(x)+y^T(Ax-b) & =\sum_1^NL_i(x_i,y)                                     \\
                 & =\sum_1^N\left[f_i(x_i)+y^TA_ix_i-{1\over N}y^Tb\right]
\end{align}
\textit{Note: a separable equation is any differential equation that can be
  expressed in the form:}
\begin{align}
  N(y)={dy\over dx}=M(x)
\end{align}
\begin{tabular*}{12cm}[p]{c c}
  \textbf{Decomposed dual problem}&
  \parbox[p]{6cm}{
    \begin{align}
      \label{eq:decomposed_dual}
      \boxed{x^{k+1}=argmin~L_i(x_i,y^k)}
    \end{align}
  }
\end{tabular*}
\\
\begin{tabular*}{12cm}[p]{c c}
  \textbf{N dual problems in parallel}&
  \parbox[p]{6cm}{
    \begin{align}
      \label{eq:n_dual}
      \boxed{x^{k+1}=argmin~L_i(x_i,y^k)}
    \end{align}
  }
\end{tabular*}
\section{Augmented Lagrangians}
When \eqref{eq:decomposed_dual} and \eqref{eq:n_dual} are not strictly convex,
a penalty term $\rho$ can be added as a constraint to guide the solution
(across iterations) to an optimized result.  A good way to think about the penalty
terms is as a force that pushes the gradient in a certain direction.
\begin{formula}{Augmented Lagrangian}
  \inlineequation{
  L_\rho(x,y)=f(x)+y^T(Ax-b)+{1\over 2}\rho||Ax-b||^2_2
  ~~~~~~~~~~~~~~~~~~~~~~~~~~~~~~~~~~~~~~}
\end{formula}
\myparbox{
  \textbf{My Thoughts:}\newline
  In response to Strang's claim \begin{quotation}
    \textquotedblleft
    The Lagrangian $L_\rho$ is not separable even if $f(x)$ is separable!
    \textquotedblright
  \end{quotation}
  Notice how before the penalty term was added, $x$ was broken up into $N$
  smaller \textit{subvectors} $x_1,...,x_N$ \refeq{eq:nseparable}.  These don't
  refer to individual elements in $x$, rather $x_i$ is itself a vector with
  \textit{at least} one entry.  The penalty \textit{cannot} break $x$ up into
  the same $N$ pieces, because each squared term in in the sum requires the
  full vector $x$.
  \begin{align}
    \label{eq:notnseparable}
    ||Ax-b||^2_2=\sum\limits_{i=1}^m (a_i^Tx)^2
  \end{align}
}
\end{document}
