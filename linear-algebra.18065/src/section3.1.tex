\documentclass{article}
\usepackage{array}
\usepackage{amsmath}
\usepackage{amssymb}
\usepackage{mathtools}
\usepackage[T1]{fontenc}
\usepackage{stackrel}
\usepackage[colorlinks]{hyperref}
\usepackage[user,titleref]{zref}
\usepackage[dvipsnames,definecolor]{xcolor}
\usepackage[most]{tcolorbox}
\usepackage{scrextend}
\usepackage{lipsum}
\usepackage{enumitem}
\usepackage{framed}
\usepackage{multicol}
\usepackage{array}
\usepackage{amsmath}
\usepackage{amssymb}
\usepackage{mathtools}
\usepackage[T1]{fontenc}
\usepackage{stackrel}
\usepackage[colorlinks]{hyperref}
\usepackage[user,titleref]{zref}
\usepackage[dvipsnames,definecolor]{xcolor}
\usepackage[most]{tcolorbox}
\usepackage{scrextend}

\usepackage{enumitem}
\usepackage{framed}
\usepackage{multicol}

\usepackage{xcolor}
\colorlet{linkequation}{red}
\colorlet{linktheo}{green}
\usepackage[colorlinks]{hyperref}

\definecolor{propbg}{RGB}{180,220,255}
\definecolor{factbg}{RGB}{170,232,145}
\definecolor{formula_header}{RGB}{30,40,210}

\newcounter{theoremc}
\newcommand{\theoremc}{\refstepcounter{theoremc}{\noindent\textbf{Theorem} \arabic{theoremc}}}
\newcommand{\myparbox}{\parbox[p]{12cm}}
\newcommand{\vpush}{\vspace{3mm}}
\newcommand{\sectionbreak}{\makebox[\linewidth]{\rule{12cm}{0.4pt}}}
\newcommand{\minmax}[2]{\min\limits_{#1}~\max\limits_{#2}}
\newcommand{\maxmin}[2]{\max\limits_{#1}~\min\limits_{#2}}

\newenvironment{note}[1][]{
  \textbf{#1}
  \begin{flushleft}
  \begin{addmargin}[2em]{0em}
}{
  \end{addmargin}
  \end{flushleft}
}
\newenvironment{proof}
{
\noindent
\begin{addmargin}[2em]{0em}
\textbf{Proof}
}
{
\end{addmargin}
}

\newenvironment{headered_note}[1][]{
\begin{flushleft}
\begin{addmargin}[2em]{0em}
\textbf{#1\\}
\vpush
\begin{addmargin}[2em]{0em}
}
{
\end{addmargin}
\end{addmargin}
\end{flushleft}
}

\newenvironment{fact}[1][]{
\begin{flushleft}
\begin{addmargin}[2em]{0em}
\textbf{Fact: #1\\}
\vpush
\begin{addmargin}[2em]{0em}
}
{
\end{addmargin}
\end{addmargin}
\end{flushleft}
}

\newenvironment{theorem}[1][]{
\begin{tcolorbox}[colback=green!30,%gray background
colframe=black,% black frame colour
% 5cm,% Use 5cm total width,
arc=3mm, auto outer arc,
]
\theoremc: \space \textbf{#1}
\begin{addmargin}[2em]{0em}
}
{
\end{addmargin}
\end{tcolorbox}}

\newenvironment{formula}[1]{
\begin{tcolorbox}[colback=white,%gray background
colframe=black,% black frame colour
% 5cm,% Use 5cm total width,
arc=3mm, auto outer arc,
]
\begin{flushleft}
{\color{formula_header} \textbf{Formula}}
\textbf{#1}
\end{flushleft}
\begin{addmargin}[2em]{0em}
}
{
\end{addmargin}
\end{tcolorbox}}

\newenvironment{headered_box}[1]{
\begin{tcolorbox}[colback=white,%gray background
colframe=black,% black frame colour
% 5cm,% Use 5cm total width,
arc=3mm, auto outer arc,
]
\begin{flushleft}
\textbf{#1}
\end{flushleft}
\begin{addmargin}[2em]{0em}
}
{
\end{addmargin}
\end{tcolorbox}}

\newenvironment{red_rounded_box}{
\begin{tcolorbox}[enhanced jigsaw,
    colback=gray!30,%gray background
    colframe=Maroon,% black frame colour
    width=5cm,% Use 5cm total width,
    arc=3mm, auto outer arc,
    boxrule=5pt,
    drop shadow={Maroon!50!gray!80}
  ]
}
{ \end{tcolorbox} }

\newcommand*{\reftheo}[1]{%
  \begingroup
    \hypersetup{
      linkcolor=linktheo,
      linkbordercolor=linktheo,
    }%
    (\ref{#1})%
  \endgroup
}

\newcommand*{\refequation}[1]{%
  \begingroup
    \hypersetup{
      linkcolor=linkequation,
      linkbordercolor=linkequation,
    }%
    \ref{#1}%
  \endgroup
}

\newenvironment{axiom}{
  \begin{tcolorbox}[enhanced jigsaw,
    colback={blue!20},
    colframe=Maroon,
    arc=2mm, auto outer arc,
    boxrule=3pt,
    drop shadow={blue!60!gray!60}
  ]
}{
  \end{tcolorbox}
}
\newenvironment{proposition}[2]{
\noindent\fcolorbox{white}{propbg}{%
  \parbox{\textwidth}{%
    \textbf{Proposition #1}
    {#2}
  }
}
}

\newcounter{definitionno}
\newenvironment{definition}[1][]{
  \begin{flushleft}
  \begin{addmargin}[2em]{0em}
  \text{Definition \refstepcounter{definitionno}\arabic{definitionno}}
  \textbf{#1}\\
  \vpush
  \begin{addmargin}[2em]{0em}
}{
  \end{addmargin}
  \end{addmargin}
  \end{flushleft}
}

\newcounter{sectiondefno}
\newenvironment{sectiondef}[1]{
  \begin{flushleft}
  \begin{addmargin}[2em]{0em}
  \textbf{Definition \thesection.\refstepcounter{sectiondefno}\arabic{sectiondefno}}
  \textbf{#1}\\
  \vpush
  \begin{addmargin}[2em]{0em}
}{
  \end{addmargin}
  \end{addmargin}
  \end{flushleft}
}

\makeatletter
\newcommand*{\inlineequation}[2][]{%
  \begingroup
    % Put \refstepcounter at the beginning, because
    % package `hyperref' sets the anchor here.
    \refstepcounter{equation}%
    \ifx\\#1\\%
    \else
      \label{#1}%
    \fi
    % prevent line breaks inside equation
    \relpenalty=10000 %
    \binoppenalty=10000 %
    \ensuremath{%
      % \displaystyle % larger fractions, ...
      #2%
    }%
    ~\@eqnnum
  \endgroup
}
\makeatother

\setlength{\fboxsep}{12pt}


\title{Linear Algebra 18.065: Changes in $A^{-1}$}
\date{2021-02-04}
\author{John Colvin}

\begin{document}
\maketitle
\begin{fact}
  \text{The inverse of $M=I-uv^T$ is}
  \begin{equation}
    M^{-1}=I+{uv^T\over 1-v^Tu}
  \end{equation}
\end{fact}
% &=I-uv^T+{uv^T\over 1-v^Tu}-u\left(I+{v^Tu\over 1-v^Tu}\right)v^T
\begin{proof}
  \begin{flushleft}
    \begin{align*}
      M(M^{-1}) & =(I-uv^T)\left(I+{uv^T \over 1-v^Tu}\right) \\
                & =I-uv^T+{(I-uv^T)uv^T\over 1-v^Tu)}         \\
      % &=I-uv^T+\left[{(I-uv^T)uv^T\over 1-v^Tu)}={uv^T-uv^Tuv^T\over 1-v^Tu}\right]\\
                & =I-uv^T+{uv^T-uv^Tuv^T\over 1-v^Tu}         \\
                & =I-uv^T+{u(I-v^Tu)v^T\over 1-v^Tu}          \\
                & =I-uv^T+uv^T                                \\
                & =I
    \end{align*}
  \end{flushleft}
\end{proof}

\begin{axiom}
  Extended Matrix $E=
    \begin{bmatrix}
      I   & u \\
      v^T & 1
    \end{bmatrix}$
  has determinant $D=1-v^Tu$
\end{axiom}
\begin{proof}
  Recall that the determinant of an $nxn$ square matrix is the sum of the $n!$
  terms that are each products $a_{1,i_1}a_{2,i_2}...a_{n,i_n}$ where
  $i_{1}\neq i_{2}\neq ... \neq i_{n}$ (each multiplicand comes from a unique row and column).  All but one term that involve a
  particular multiplicand $u_i$ are $0$ (because the matrix is sparse, and has
  all zeros except along the main diagonal, the last row, and the last column).
  The only term with such a multiplcand that's $\neq 0$ is the one that takes diagnonal
  entry from every row, except on row $i$ and the last row.  If the multiplicand $u_i$
  is fixed, and every other row chooses $a_{j,j}$ (the $1$ on the diagnonal), the last row
  has only one candidate left: $v^T_i$.  Of course, the $1$'s along the diagnonal form
  a non-zero term as well: $1$.  Therefore
  \begin{equation}
    \det{E}=1-v^Tu
  \end{equation}
  \begin{align*}
    E_4 & =
    \begin{bmatrix}
      1     & 0     & 0     & u_1 \\
      0     & 1     & 0     & u_2 \\
      0     & 0     & 1     & u_3 \\
      v^T_1 & v^T_2 & v^T_3 & 1
    \end{bmatrix}
  \end{align*}
  Find $E_4^{-1}$ through elimination. Subtract $v^T$ from the last row to
  cancel it out
  \begin{align*}
    \stackrel{Elim}{
      \begin{bmatrix}
        1      & 0      & 0      & u_1 \\
        0      & 1      & 0      & u_2 \\
        0      & 0      & 1      & u_3 \\
        -v_1^T & -v_2^T & -v_3^T & 1
      \end{bmatrix}
    }
    \stackrel{E}{
      \begin{bmatrix}
        1     & 0     & 0     & u_1 \\
        0     & 1     & 0     & u_2 \\
        0     & 0     & 1     & u_3 \\
        v^T_1 & v^T_2 & v^T_3 & 1
      \end{bmatrix}
    }
    =
    \begin{bmatrix}
      1 & 0 & 0 & u_1    \\
      0 & 1 & 0 & u_2    \\
      0 & 0 & 1 & u_3    \\
      0 & 0 & 0 & 1-v^Tu
    \end{bmatrix}
    =
    \begin{bmatrix}
      I & u \\
      0 & D
    \end{bmatrix} \\
    \implies
    E_4^{-1}=(Elim)^{-1}
    \begin{bmatrix}
      I & u \\
      0 & D
    \end{bmatrix}
  \end{align*}
\end{proof}
\begin{formula}{Sherman-Morrison-Woodbury}
  \begin{align}
    M      & =A-UV^T\implies\nonumber                   \\
    M^{-1} & =A^{-1}+A^{-1}U(I-V^TA^{-1})^{-1}V^TA^{-1}
  \end{align}
\end{formula}

\begin{proof}
  \begin{tcolorbox}[
      colback=white,colframe=black,width=14cm,arc=3mm, auto outer arc]
    \begin{align*}
      \begin{bmatrix}
        I          & 0 \\
        -V^TA^{-1} & I
      \end{bmatrix}
      \begin{bmatrix}
        A   & U \\
        V^T & I
      \end{bmatrix}
       & =
      \begin{bmatrix}
        A & U            \\
        0 & I-V^TA^{-1}U
      \end{bmatrix} \\
      And                        \\
      \begin{bmatrix}
        A & U            \\
        0 & I-V^TA^{-1}U
      \end{bmatrix}^{-1}
       & =
      \begin{bmatrix}
        A^{-1} & -A^{-1}U(I-V^TA^{-1}U)^{-1} \\
        0      & (I-V^TA^{-1}U)^{-1}
      \end{bmatrix}
      \\ \implies\\
      \begin{bmatrix}
        A   & U \\
        V^T & I
      \end{bmatrix}^{-1}
       & =
      \begin{bmatrix}
        A^{-1} & -A^{-1}U(I-V^TA^{-1}U)^{-1} \\
        0      & (I-V^TA^{-1}U)^{-1}
      \end{bmatrix}
      \begin{bmatrix}
        I          & 0 \\
        -V^TA^{-1} & I
      \end{bmatrix} \\
       & =
      \begin{bmatrix}
        A^{-1}+A^{-1}UC^{-1}V^TA^{-1} & -A^{-1}UC^{-1} \\
        -C^{-1}V^TA^{-1}              & C^{-1}
      \end{bmatrix}
    \end{align*}
  \end{tcolorbox}
  \begin{tcolorbox}[
      colback=white,colframe=black,width=14cm,arc=3mm, auto outer arc]
    \begin{align*}
      \begin{bmatrix}
        I & -U \\
        0 & I
      \end{bmatrix}
      \begin{bmatrix}
        A   & U \\
        V^T & I
      \end{bmatrix}
       & =
      \begin{bmatrix}
        A-UV^T & 0 \\
        V^T    & I
      \end{bmatrix} \\
      \implies
      \begin{bmatrix}
        M^{-1}     & 0 \\
        -V^TM^{-1} & I
      \end{bmatrix}
      \begin{bmatrix}
        M   & 0 \\
        V^T & I
      \end{bmatrix}
       & =I                      \\
      \implies
      \begin{bmatrix}
        M^{-1}     & 0 \\
        -V^TM^{-1} & I
      \end{bmatrix}
      \begin{bmatrix}
        I & -U \\
        0 & I
      \end{bmatrix}
       & =
      \begin{bmatrix}
        M^{-1}     & -M^{-1}U     \\
        -V^TM^{-1} & V^TM^{-1}U+I
      \end{bmatrix} \\
      \implies
      \begin{bmatrix}
        A^{-1}+A^{-1}UC^{-1}V^TA^{-1} & -A^{-1}UC^{-1} \\
        -C^{-1}V^TA^{-1}              & C^{-1}
      \end{bmatrix}
       & =
      \begin{bmatrix}
        M^{-1}     & -M^{-1}U     \\
        -V^TM^{-1} & V^TM^{-1}U+I
      \end{bmatrix}
    \end{align*}
    The (1,1) blocks imply
    \begin{align}
      \implies
      M^{-1}=A^{-1}+A^{-1}UC^{-1}V^TA^{-1} & -A^{-1}UC^{-1}
    \end{align}
  \end{tcolorbox}
\end{proof}

\begin{axiom}
  \textbf{Key Identities}
  \begin{equation}
    B(I_m+AB)=(I_n+BA)B
  \end{equation}
  \begin{equation}
    \label{eq:key_identity_2}
    B(I_m+AB)^{-1}=(I_n+BA)^{-1}B
  \end{equation}
  \begin{equation}
    A^T(AA^T+\lambda I_n)^{-1}=(A^TA+\lambda I_m)^{-1}A^T
  \end{equation}
  \begin{equation}
    U(I_k-V^TU)=(I_n-UV^T)U
  \end{equation}
  \begin{equation}
    B^{-1}-A^{-1}=B^{-1}(A-B)A^{-1}
  \end{equation}
\end{axiom}

\begin{fact}
  If $A$ is $mxn$ and $B$ is $nxm$ then \eqref{eq:key_identity_2} holds true because
  $AB$ and $BA$ have the same eigenvalues.  This means that $AB$ is invertible exactly when
  $BA$ is invertible ($I+AB$ and $I+BA$ are not invertible when $\lambda=-1$, in which $0$
  would be an eigenvalue of both)\\ \\
  \begin{proof}
    \begin{align*}
      ABx    & =\lambda x    \\
      B(ABx) & =B(\lambda x) \\
      (BA)Bx & =\lambda Bx
    \end{align*}
  \end{proof}
\end{fact}

\begin{axiom}
  \textbf{The Kalman Filter}\\ \\
  \begin{tabular}{|m{4.5cm}|m{6cm}}
    \hline \\
    \parbox[t]{5cm}{
      1. Original ($A~0$)             \\
      2. State Update ($-I~I$)        \\
      3. Measurement ($0~r$)
    }&
    \parbox[t]{6cm}{
      $A_{new}=
        \begin{bmatrix}
          A  & 0 \\
          -I & I \\
          0  & r
        \end{bmatrix}
        \begin{bmatrix}
          x_{old} \\
          x_{new}
        \end{bmatrix}~~~~=
        \begin{bmatrix}
          b         \\
          v\Delta t \\
          b_{m+1}
        \end{bmatrix}$
    }                               \\ \\
    &
    $~~~~~~~=\begin{bmatrix}
      Ax_{old}\\
      -x_{old}\\
      0
    \end{bmatrix}+
    \begin{bmatrix}
      0\\
      x_{new}\\
      rx_{new}
    \end{bmatrix}=
    \begin{bmatrix}
      b\\
      v\Delta t\\
      b_{m+1}
    \end{bmatrix}$
    \\
    \hline
    \textbf{K=Kalman gain matrix} &
    \begin{equation}
      \label{eq:kalman_update}
      \hat{x}_{new}=\hat{x}_{state}+K(b_{m+1}-r\hat{x}_{state})
    \end{equation}
    \\
    \hline
  \end{tabular}
  \\ \\
  \parbox[m]{10cm}{
    \begin{itemize}
      \item Note that $r$ is a \textit{row vector} with the same number of columns as $A$
      \item Instead of simply adding the new equation to $Ax=b$, two equations are added: the new measurement,
      and the prediction.
      \item Both new equations are part of the projection source, and the solution to $A^TA\hat{x}=b$
      minimizes the magnitude of the error $e=b-A^Tb$ 
    \end{itemize}
    The equation above \eqref{eq:kalman_update} integrates the update in 2 components:
    \begin{enumerate}[nolistsep]
      \item Update via recent measurements $rx_{new}=b_{m+1}$
      \item Update via estimation from prior measurement (as a function of the previous state) $x_{new}-x_{old}=v\Delta t$
    \end{enumerate}
    Since neither is assumed to be $100\%$ accurate, a \textit{weighted avg} of the 2 updates is incorporated into
    $\hat{x}_{new}$.  If the measurements agree with one another, then $b_{m+1}=2$
  }
\end{axiom}

\begin{axiom}
  \textbf{Quasi-Newton Update Methods}\\
  \begin{itemize}
    \item Used to find approx solutions to \begin{equation}\label{eq:quasi_newton_fxzero}f(x)=0\end{equation}
    \item \eqref{eq:quasi_newton_condition} leads to \eqref{eq:quasi_newton_xnew} as $f(x_{new})\approx 0$ (mentioned above)
  \end{itemize}
  \begin{align}
    J_{ik}&={\partial{f_i}\over\partial{x_k}}\\
    \label{eq:quasi_newton_condition}
    J_{new}(x_{new}-x_{old})&=f_{new}-f_{old}\\
    \label{eq:quasi_newton_xnew}
    x_{new}&=x_{old}-J(x_{old})^{-1}f(x_{old})
  \end{align}
  \begin{proof}
    \begin{align*}
      \eqref{eq:quasi_newton_condition}&\implies x_{new}-x_{old}=J(f_{new}-f_{old})^{-1}\\
      \eqref{eq:quasi_newton_fxzero}&\implies x_{new}-x_{old}=J(-f_{old})^{-1}\\
      &\implies x_{new}=x_{old}-J(f_{old})^{-1}
    \end{align*}
  \end{proof}
\end{axiom}

\begin{fact}
  When the solution $w$ to $Aw=b$ is known, as is the solution $z$ to $Az=u$,
  the solution $x$ to $(A-uv^T)x=b$ is:
  \begin{equation}
    x=w+{v^Tw\over 1-v^Tz}z
  \end{equation} 
  \begin{proof}
    \begin{align*}
      Ax-uv^Tx&=b\\
      Ax-Azv^Tx&=b=Aw\\
      x-zv^Tx&=w\\
      (I-zv^T)x&=w\\
      x&=(I-zv^T)^{-1}w\\
      x&=\left(I+{zv^T\over 1-v^Tz}\right)w\\
      x&=w+{v^Tw\over 1-v^Tz}z
    \end{align*}
  \end{proof}
\end{fact}
% {
%   \setlength{\fboxrule}{5pt}
%   \fcolorbox{black}{gray!30}{My text}
%   \setlength{\fboxrule}{1pt}

%   \fcolorbox{black}{gray!30}{My text}
% }
% using a tabular with \parbox inside \framebox
\noindent\begin{tabular}{p{.45\textwidth}p{.45\textwidth}}
  \framebox{\parbox[t]{.4\textwidth}{\lipsum[1]}} & \framebox{\parbox[t]{.4\textwidth}{
      \lipsum[1]
      \begin{enumerate}[nolistsep]
        \item box2-item1
        \item box2-item2
        \item box2-item3
      \end{enumerate}}}
\end{tabular}

\newpage

% now, using multicol and framed
\begin{multicols}{2}
  \begin{framed}
    \lipsum[1]
  \end{framed}
  \begin{framed}
    \lipsum[1]
    \begin{enumerate}[nolistsep]
      \item box2-item1
      \item box2-item2
      \item box2-item3
    \end{enumerate}
  \end{framed}
\end{multicols}

\end{document}
