\documentclass{article}
\usepackage{amsmath}
\usepackage{changepage}
\usepackage{xcolor}
\usepackage[user,titleref]{zref}

% \begin{document}
% \newcounter{set}
% \setcounter{set}{1}
% \newcounter{problem}[set]

% \newcommand{\problem}{\refstepcounter{problem}{\vspace{2\baselineskip}\noindent\large \bfseries Problem~\arabic{set}.\arabic{problem}}\\}

% \problem
% \textit{Sum-product algorithm:}  Consider the sum-product\ldots.

% \problem
% \textit{Max-marginals:} Consider the max-marginals\ldots.

% \stepcounter{problem}
% \problem
% Demonstraction of \verb"\stepcounter"

% \addtocounter{problem}{-1}
% \problem
% \end{document}

\newcommand{\exampleIndent}{20pt}
\newcommand{\solutionIndent}{20pt}

\newcommand{\myparagraph}[1]{\paragraph{#1}\mbox{}\\}
\newcommand{\header}[1]{
  \begin{flushleft}
    \large
    \textbf{#1}
  \end{flushleft}
}
\newcommand{\problem}[1]{
  \begin{flushleft}
    \large
    \textbf{#1.}
  \end{flushleft}
}

\newenvironment{solution}[1] {
\begin{flushleft}
  \large
  \textbf{Solution}
\end{flushleft}
\begin{adjustwidth}{\solutionIndent}{2pt} {
    {#1}
  }
\end{adjustwidth}
}

\begin{document}
\begin{enumerate}
  \item
        \begin{enumerate}
          \item How many different 7-place license plates are possible if the first 2 places
                are for letters and the other 5 for numbers?
                \begin{solution} {
                    as letters/numbers may be repeated, there are $26^{2}(10^{5})$ possible license plates
                  }
                \end{solution}
          \item Repeat part (a) under the assumption that no letter or numbers
                can be repeated in a single license plate.
                \begin{solution} {
                    there are $26 \choose 2$ possibilities for the first 2 places (letters), and
                    $10 \choose 5$ for the remaining 5 places (numbers).  Thus the answer is
                    \begin{align*}
                      \frac{26!}{24!}\frac{10!}{5!}
                    \end{align*}
                  }
                \end{solution}
        \end{enumerate}
  \item How many outcome sequences are possible when a die is rolled four times, where
        we say, for instance, that the outcome is 3, 4, 3, 1 if the first roll landed on 3,
        the second on 4, the third on 3, and the fourth on 1?
        \begin{solution} {
            there are $6^{4}$ possibilities (duh)
          }
        \end{solution}
  \item Twenty workers are to be assigned to 20 different jobs, one to each job.
        How many different assignments are possible?
        \begin{solution} {
            This is simply the number of permutations of the 20 workers, as there
            is a 1:1 mapping of worker:job.
            \begin{align*}
              20!
            \end{align*}
          }
        \end{solution}
  \item John, Jim, Jay, and Jack have formed a band consisting of 4 instruments.
        \begin{enumerate}
          \item If each of the boys can play all 4 instruments, how many different arrangements are possible?
                \begin{solution} {
                    If each can play all 4, then the question is identitical to (3).  The answer is
                    \begin{align*}
                      4!=24
                    \end{align*}
                  }
                \end{solution}
          \item What if John and Jim can play all 4 instruments, but Jay and Jack can each play only piano and drums?
                \begin{solution} {
                    If only 2 of the players can play all instruments, and the other 2 can play 2 instruments,
                    the answer does NOT become
                    \begin{align*}
                      \frac{4!}{2!}(2!)=4!=24
                    \end{align*}
                    This is what I thought at first, but then it occurred to me, that the fact
                    that 2 of the players can play all 4 instruments is irrelevant if the other
                    2 players can only play 2 instruments.
                    The former 2 players MUST play the 2 instruments that the latter 2 can't play.
                    Therefore, the answer actually becomes
                    \begin{align*}
                      2!(2!)=4
                    \end{align*}
                  }
                \end{solution}
        \end{enumerate}
  \item For years, telephone area codes in the United States and Canada
        consisted of a sequence of three digits. The first digit was an integer
        between 2 and 9, the second digit was either 0 or 1, and the third digit
        was any integer from 1 to 9.
        \begin{enumerate}
          \item How many area codes were possible?
          \begin{solution} {
            There were 8 possibilities for the first number, 2 for the second, and 9 for the third.
            \begin{align*}
              (8)(2)(9)
            \end{align*}
            possibilities altogether.
          }
          \end{solution}
          \item How many area codes starting with a 4 were possible?
          \begin{solution} {
            The number of choices for the first digit changes from 8 to 1, leaving
            \begin{align*}
              (1)(2)(9)=18
            \end{align*}
            possibilities.
          }
          \end{solution}
        \end{enumerate}
    \item A well-known nursery rhyme starts as follows:
    \begin{quotation}
    As I was going to St. Ives\\
    I met a man with 7 wives.\\
    Each wife had 7 sacks.\\
    Each sack had 7 cats.\\
    Each cat had 7 kittens.\\
    \end{quotation}
    How many kittens did the traveler meet?
    \begin{solution} {
      Isn't it obvious? $7*7*7*7=7^{4}$
    }
    \end{solution}
    \item
    \begin{enumerate}
      \item In how many ways can 3 boys and 3 girls sit in a row?
      \begin{solution} {
        It seems the gender is irrelevant to this question.  The more direct question is, "how many ways can 6 people sit together?"
        \begin{align*}
          6!
        \end{align*}
      }
      \end{solution}
      \item In how many ways can 3 boys and 3 girls sit in a row if the boys
      and the girls are each to sit together?
      \begin{solution}
        There are $3!$ ways to arrange the boys amongst themselves, and the same number of ways to arrange the girls amongst themselves.
        There are also $2!$ ways to position the block of boys and girls relative to each other.
        \begin{align*}
          (3!)(3!)=36
        \end{align*}
      \end{solution}
      \item In how many ways if only the boys must sit together?
      \begin{solution} {
        The boys can be arranged in $3!$ ways amongst themselves.  Since they must sit together,
        they collectively act as a 4th person with whom to permute seating arrangements with the 3 girls.
        Like each individual girl, there are 4 possible positions in which the block of 3 boys can sit.
        Therefore, there are
        \begin{align*}
          (3!)(4!)=144
        \end{align*}
        combinations.
      }
      \end{solution}
      \item In how many ways if no two people of the same sex are allowed
      to sit together?
      \begin{solution} {
        A simple way to view the problem, is to fix the girls positions, and then insert the boys between them.
        2 of the 3 boys must sit between the 1st and 2nd, and 2nd and 3rd girls.
        The only remaining choice is whether to put the 3rd boy in front of the 1st girl, or after the last one. 
        So the answer is simply:
        \begin{align*}
          (permutations_{girls})(permutations_{boys})(shifts_{boys})=3!(3!)2
        \end{align*}
        A more general way to view the problem, is in terms of $n$ children, of which $0<m<n$ are girls.
        The general solution calls for counting the number of Nonnegative Integer Vector Solutions.
        \mbox{}\\ \\
        If the $m$ girls' positions are fixed, there are $m+1$ groups in which to place the $n-m$ boys.
        The first group represents the position before the first girl, the last group represents
        the position after the last girl, and the middle groups are each between 2 girls.
        \mbox{}\\ \\
        Let $\vec{x}=({x_{1},x_{2},...,x_{m+1}})$ represent the number of boys in each group.
        Note that the solutions have the following constraints:
        \begin{align*}
          &x_{1}\geq 0\\
          &x_{i}\geq 1~\{i~|~2\leq i \leq m\}\\
          &x_{m+1}\geq 0
        \end{align*}
        With a simple substiution, combinatorial logic can be used to count the number of solutions
        that satisfy the constraints.  Let
        \begin{align*}
          \vec{y}=(x_{1}+1,x_{2},x_{3},...,x_{m},x_{m+1}+1)
        \end{align*}

        Instead of counting solutions to $\sum{x_{i}}=n-m$, count solutions to $\sum{y_{i}}=n-m+2$.
        The number of solutions to the latter is
        \begin{align*}
          n-m+1 \choose m
        \end{align*}
        \mbox{}\\ \\
        Don't lose sight of the original problem!  Remember, the number of solutions to the equation above
        does not account for the number of permutations of the girls and boys.  So the count must be multiplied
        by $(n-m)!(m!)$ to get the right answer.
        If $num~children=n=6$ and $num~girls=m=3$, then with the general formula, the number of ways in which
        the children may be seated, such that there are no 2 consecutive children of the same gender, is:
        \begin{align*}
          (3!)(3!){4 \choose 3}=(3!)(3!)\frac{4!}{1!(3!)}=(3!)(3!)4
        \end{align*}
        Conflabit! What happened? This is the number of ways the children could be arranged so that no 2 \textit{girls} sit next to eachother.
        However, this model does NOT discount solutions where 2 \textit{boys} sit together.  Hmmm, how to get
        rid of those pesky solutions in which 2 boys are adjacent...I got it!  We'll just count those solutions,
        and subtract them from the total.  How hard can that be?  Well, too hard for now!  To be continued...
      }
      \end{solution}
    \end{enumerate}
    \item When all letters are used, how many different letter arrangements
    can be made from the following letters:
    \begin{enumerate}
      \item Fluke
      \begin{solution} {
        $5!=120$ (duh)
      }
      \end{solution}
      \item Propose
      \begin{solution} {
        Assume permutations are case-insensitive.  Take the number of
        permutations of the letters as if they were distinguishable, and divide by
        the number of permutations of the repeated letters.
        In this case, each permutation has 2 p's and 2 o's. Therefore, there are
        2 permutations of o's and 2 permutations of p's per distinguishable permutation.
        \begin{align*}
          \frac{7!}{(2!)(2!)}=\frac{7!}{4}=7*6*5*3*2=1260
        \end{align*}
      }
      \end{solution}
      \item Mississippi
      \begin{solution} {
        Wow, okay.  Is there any word in the English language that produces a greater ratio of (total-distinct):total letters?
        There are 11 letters altogether, a measley one of which is NOT repeated: 4 s's, 4 p's, 2 i's, and 1 sad, lonely M.
        \begin{align*}
          \frac{11!}{4!4!2!}=34650
        \end{align*}
      }
      \end{solution}
      \item Arrange
      \begin{solution} {
        \begin{align*}
          \frac{7!}{2!2!}=1260
        \end{align*}
      }
      \end{solution}
    \end{enumerate}
    \item A child has 12 blocks, of which 6 are black, 4 are red, 1 is white, and
    1 is blue. If the child puts the blocks in a line, how many arrangements
    are possible?
    \begin{solution} {
      This answer can be obtained with the same formula used to count distinct permutations of letters.
      The letters, and blocks are both symbols.  The number of permutations of a set of symbols is equal to
      the set size factorial, divided by the number of permutations of repeated symbols.  It could also be
      viewed as divided by the number of permutations of each symbol in the set. As non-repeated symbols
      would result in a division by 1, they would not change the count.
      \begin{align*}
        \frac{12!}{6!4!}=27720
      \end{align*}
    }
    \end{solution}
    \item In how many ways can 8 people be seated in a row if
    \begin{enumerate}
      \item there are no restrictions on the seating arrangement?
      \begin{solution} {
        8!
      }
      \end{solution}
      \item persons A and B must sit next to each other?
      \begin{solution} {
        One way to look at this is to consider A and B 1 person, then count the permutations of
        this slightly smaller set (with one less person), and multiply that result by the number of
        permutations which A and B could have amongst themselves.
        \begin{align*}
          7!2!
        \end{align*}
      }
      \end{solution}
      \item there are 4 men and 4 women and no 2 men or 2 women can sit
      next to each other?
      \begin{solution} {
        Didn't we do this already?
        \begin{align*}
          4!4!2
        \end{align*}
      }
      \end{solution}
      \item there are 5 men and they must sit next to one another?
      \begin{solution} {
        There are $5!$ ways to permute the men amongst themselves, $3!$ ways to permute the women,
        and 4 positions in which the block of men may be inserted (relative to the women).
        \begin{align*}
          5!3!4=2880
        \end{align*}
      }
      \end{solution}
      \item there are 4 married couples and each couple must sit together?
      \begin{solution} {
        There are $4!$ ways to permute the couples independent of their internal arrangements,
        and of course 2 ways to permute each couple internally, which means $2^4$ internal permutations
        per permutation of the set of couples.
        \begin{align*}
          4!2^4=384
        \end{align*}
      }
      \end{solution}
    \end{enumerate}
    \item In how many ways can 3 novels, 2 mathematics books, and 1
    chemistry book be arranged on a bookshelf if
\end{enumerate}
\end{document}