\documentclass{article}
\usepackage{array}
\usepackage{amsmath}
\usepackage{amssymb}
\usepackage{mathtools}
\usepackage{stackrel}
\usepackage[dvipsnames,definecolor]{xcolor}
\usepackage{array}
\usepackage{amsmath}
\usepackage{amssymb}
\usepackage{mathtools}
\usepackage[T1]{fontenc}
\usepackage{stackrel}
\usepackage[colorlinks]{hyperref}
\usepackage[user,titleref]{zref}
\usepackage[dvipsnames,definecolor]{xcolor}
\usepackage[most]{tcolorbox}
\usepackage{scrextend}

\usepackage{enumitem}
\usepackage{framed}
\usepackage{multicol}

\usepackage{xcolor}
\colorlet{linkequation}{red}
\colorlet{linktheo}{green}
\usepackage[colorlinks]{hyperref}

\definecolor{propbg}{RGB}{180,220,255}
\definecolor{factbg}{RGB}{170,232,145}
\definecolor{formula_header}{RGB}{30,40,210}

\newcounter{theoremc}
\newcommand{\theoremc}{\refstepcounter{theoremc}{\noindent\textbf{Theorem} \arabic{theoremc}}}
\newcommand{\myparbox}{\parbox[p]{12cm}}
\newcommand{\vpush}{\vspace{3mm}}
\newcommand{\sectionbreak}{\makebox[\linewidth]{\rule{12cm}{0.4pt}}}
\newcommand{\minmax}[2]{\min\limits_{#1}~\max\limits_{#2}}
\newcommand{\maxmin}[2]{\max\limits_{#1}~\min\limits_{#2}}

\newenvironment{note}[1][]{
  \textbf{#1}
  \begin{flushleft}
  \begin{addmargin}[2em]{0em}
}{
  \end{addmargin}
  \end{flushleft}
}
\newenvironment{proof}
{
\noindent
\begin{addmargin}[2em]{0em}
\textbf{Proof}
}
{
\end{addmargin}
}

\newenvironment{headered_note}[1][]{
\begin{flushleft}
\begin{addmargin}[2em]{0em}
\textbf{#1\\}
\vpush
\begin{addmargin}[2em]{0em}
}
{
\end{addmargin}
\end{addmargin}
\end{flushleft}
}

\newenvironment{fact}[1][]{
\begin{flushleft}
\begin{addmargin}[2em]{0em}
\textbf{Fact: #1\\}
\vpush
\begin{addmargin}[2em]{0em}
}
{
\end{addmargin}
\end{addmargin}
\end{flushleft}
}

\newenvironment{theorem}[1][]{
\begin{tcolorbox}[colback=green!30,%gray background
colframe=black,% black frame colour
% 5cm,% Use 5cm total width,
arc=3mm, auto outer arc,
]
\theoremc: \space \textbf{#1}
\begin{addmargin}[2em]{0em}
}
{
\end{addmargin}
\end{tcolorbox}}

\newenvironment{formula}[1]{
\begin{tcolorbox}[colback=white,%gray background
colframe=black,% black frame colour
% 5cm,% Use 5cm total width,
arc=3mm, auto outer arc,
]
\begin{flushleft}
{\color{formula_header} \textbf{Formula}}
\textbf{#1}
\end{flushleft}
\begin{addmargin}[2em]{0em}
}
{
\end{addmargin}
\end{tcolorbox}}

\newenvironment{headered_box}[1]{
\begin{tcolorbox}[colback=white,%gray background
colframe=black,% black frame colour
% 5cm,% Use 5cm total width,
arc=3mm, auto outer arc,
]
\begin{flushleft}
\textbf{#1}
\end{flushleft}
\begin{addmargin}[2em]{0em}
}
{
\end{addmargin}
\end{tcolorbox}}

\newenvironment{red_rounded_box}{
\begin{tcolorbox}[enhanced jigsaw,
    colback=gray!30,%gray background
    colframe=Maroon,% black frame colour
    width=5cm,% Use 5cm total width,
    arc=3mm, auto outer arc,
    boxrule=5pt,
    drop shadow={Maroon!50!gray!80}
  ]
}
{ \end{tcolorbox} }

\newcommand*{\reftheo}[1]{%
  \begingroup
    \hypersetup{
      linkcolor=linktheo,
      linkbordercolor=linktheo,
    }%
    (\ref{#1})%
  \endgroup
}

\newcommand*{\refequation}[1]{%
  \begingroup
    \hypersetup{
      linkcolor=linkequation,
      linkbordercolor=linkequation,
    }%
    \ref{#1}%
  \endgroup
}

\newenvironment{axiom}{
  \begin{tcolorbox}[enhanced jigsaw,
    colback={blue!20},
    colframe=Maroon,
    arc=2mm, auto outer arc,
    boxrule=3pt,
    drop shadow={blue!60!gray!60}
  ]
}{
  \end{tcolorbox}
}
\newenvironment{proposition}[2]{
\noindent\fcolorbox{white}{propbg}{%
  \parbox{\textwidth}{%
    \textbf{Proposition #1}
    {#2}
  }
}
}

\newcounter{definitionno}
\newenvironment{definition}[1][]{
  \begin{flushleft}
  \begin{addmargin}[2em]{0em}
  \text{Definition \refstepcounter{definitionno}\arabic{definitionno}}
  \textbf{#1}\\
  \vpush
  \begin{addmargin}[2em]{0em}
}{
  \end{addmargin}
  \end{addmargin}
  \end{flushleft}
}

\newcounter{sectiondefno}
\newenvironment{sectiondef}[1]{
  \begin{flushleft}
  \begin{addmargin}[2em]{0em}
  \textbf{Definition \thesection.\refstepcounter{sectiondefno}\arabic{sectiondefno}}
  \textbf{#1}\\
  \vpush
  \begin{addmargin}[2em]{0em}
}{
  \end{addmargin}
  \end{addmargin}
  \end{flushleft}
}

\makeatletter
\newcommand*{\inlineequation}[2][]{%
  \begingroup
    % Put \refstepcounter at the beginning, because
    % package `hyperref' sets the anchor here.
    \refstepcounter{equation}%
    \ifx\\#1\\%
    \else
      \label{#1}%
    \fi
    % prevent line breaks inside equation
    \relpenalty=10000 %
    \binoppenalty=10000 %
    \ensuremath{%
      % \displaystyle % larger fractions, ...
      #2%
    }%
    ~\@eqnnum
  \endgroup
}
\makeatother

\setlength{\fboxsep}{12pt}

\title{Convex Optimization Rockafeller Ch 1: Introduction}
\date{2021-02-26}
\author{John Colvin}

\begin{document}
\maketitle
\section{Basic Definitions}
\begin{definition}[Least-Squares Problems]
  A least squares problem is an optimziation with no constraints (i.e., $m=0$)
  and an \textbf{objective} which is a sum of squares of terms of the form $a_i^Tx-b_i$
  \begin{align}
    \min f_0(x)=||Ax-b||_2^2=\sum_{i=1}^k(a_i^Tx-b_i)^2
  \end{align}
  Reduces to solving a set of linear equations
  \begin{align}
    (A^TA)x=A^Tb
  \end{align}
  \begin{itemize}
    \item Known algorithms can solve least squares in time $n^2k$ ($k$ is a known constant)
    \item Desktop computer can solve a sparse least-squares problem with tens of thousands of variables,
          and hundreds of thousands of terms, in $\approx 1$ minute.
  \end{itemize}
  \begin{headered_note}[Variations]
    \textbf{Weighted Least Squares}
    \begin{align}
      \min \sum_{i=1}^k w_i(a_i^Tx-b_i)^2
    \end{align}
    \textit{My thoughts}: presumably $\sum_{i=1}^k w_i=1$\\
    \vpush
    \textbf{Regularization}
    \begin{align}
      \sum_{i=1}^k(a_i^Tx-b_i)^2+\rho \sum_{i=1}^px_i^2,~\rho>0
    \end{align}
  \end{headered_note}
\end{definition}
\begin{definition}[Linear Programming]
  The objective and all contraint functions are linear
  \begin{align}
    \min c^Tx \text{ subject to }a_i^Tx\leq b_i,~i=1,...,m
  \end{align}
  The vectors $c,a_1,...,a_m\in R^n$ and scalars $b_1,...,b_m\in R$ are
  \textbf{problem parameters} that specify the object and contstraint functions.
  \begin{itemize}
    \item No simple analytical formula for the solution of a linear program
          (unlike least squares problems)
    \item \textbf{Dantzig's Simplex Method} is a good algo
    \item More recently, \textbf{Interior point methods} have emerged as promising strategies
  \end{itemize}
\end{definition}
\begin{definition}[Chebychev Approximation Problem]
  Goal: find $x$ that produces the smallest absolute value of any $a_i^Tx-b_i$.
  $x\in R^n$ is the variable, and $a_1,...,a_k\in R^n,b_1,...,b_k\in R$ are
  parameters that specify the problem instance.  Note resemblence to \textbf{Least Squares}.
  \begin{align}
    \min~\max_{i=1,...,k}|a_i^Tx-b_i|
  \end{align}
  \vpush
  Reduces to the following (variables $x\in R^n,t\in R$)
  \begin{align}
     & \min t                                            \\
    \nonumber
     & \text{subject to } & a_i^Tx-t\leq b_i,i=1,...,k   \\
    \nonumber
     & & -a_i^Tx-t\leq -b_i,i=1,...,k
  \end{align}
\end{definition}
\begin{definition}[Convex Optimization]
  A convex optimization problem is one of the form
  \begin{align}
     & \min f_0(x)                                  \\
    \nonumber
     & \text{ subject to }f_i(x)\leq b_i,~i=1,...,m
  \end{align}
\end{definition}
\end{document}
